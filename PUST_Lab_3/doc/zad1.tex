\chapter{Zadanie 1: Punkt pracy}
Pierwszym poleceniem by�o sprawdzenie mo�liwo�ci sterowania i pomiaru w komunikacji ze stanowiskiem oraz okre�lenie warto�ci wyj�� obiektu $Y_{1pp}$ (pomiaru $T1$) oraz $Y_{2pp}$ (pomiaru $T3$) w punkcie pracy $U_{1pp} = 25 + nz$ oraz $U_{2pp} = 30 + nz$, gdzie dla naszego zespo�u $nz = 11$. Obiekt zachowywa� si� prawid�owo - umo�liwia� sterowanie temperatur� i jej odczyt. Nast�pnie przeszli�my do badania zachowania obiektu w punkcie pracy: Ustawili�my warto�� sterowania (moc grzania grza�ek $G1$ i $G3$) na $U_{1pp}$ i $U_{2pp}$ i odczekuj�c znaczn� ilo�� czasu (powy�ej 8 minut). Wyj�cia ustabilizowa�y si� w pobli�u warto�ci $39$ oraz $41.31$  (czasami wyst�powa�y niewielkie skoki spowodowane prawdopodobnie zak��ceniami z otoczenia). Ostatecznie zdecydowali�my si� zachowa� te warto�ci.


\begin{figure}[tb]
	\centering
	\begin{tikzpicture}
	\begin{axis}[
	width=0.9\textwidth,
	xmin=0,xmax=410,ymin=37,ymax=42,
	xlabel={$k$},
	ylabel={$Y_{pp}(k)$},
	xtick={0,50,100,150,200,250,300,350,400},
	ytick={37,38,39,40,41,42,44},
	y tick label style={/pgf/number format/1000 sep=},
	]
	\addplot[blue,semithick] file {wykresy/z1Y1.txt};
	\addplot[orange,semithick] file {wykresy/z1Y2.txt};
	
	\legend{$Y_{1pp}$,$Y_{2pp}$}
	\end{axis}
	\end{tikzpicture}
	\caption{Wykresy Y(k) dla dla punkt�w pracy Upp}
	\label{skok}
\end{figure}