\chapter{Zadanie 3: Znormalizowane odpowiedzi skokowe}
Przedstawione odpowiedzi skokowe na rys. \ref{normal} zosta�y wykonane przez wykonanie skoku jednostkowego na wszystkich torach oraz obci�cie pierwszych $10$ pr�bek, gdy� zmiana sterowania odbywa�a si� w $10$ kroku. Z tego wynika, �e wykresy startuj� od chwili $k=11$.  
\begin{figure}[tb]
	\centering
	\begin{tikzpicture}
	\begin{groupplot}[group style={group size=1 by 4,vertical sep={2 cm}},
	width=0.9\textwidth,height=0.25\textwidth]
	\nextgroupplot
	[
	xmin=0,xmax=250,ymin=0,ymax=2,
	xlabel={$k$},
	ylabel={$Y_1(U_1)$},
	xtick={0,250},
	ytick={0,0.5,1,1.5,2},
	y tick label style={/pgf/number format/1000 sep=},
	]
	\addplot[blue,semithick] file {wykresy/z3Y1U1.txt};
	\nextgroupplot
	[
	xmin=0,xmax=250,ymin=0,ymax=2,
	xlabel={$k$},
	ylabel={$Y_1(U_2)$},
	xtick={0,250},
	ytick={0,0.5,1,1.5,2},
	y tick label style={/pgf/number format/1000 sep=},
	legend pos=south east,
	]
	\addplot[green,semithick] file {wykresy/z3Y1U2.txt};
	\nextgroupplot
	[
	xmin=0,xmax=250,ymin=0,ymax=2,
	xlabel={$k$},
	ylabel={$Y_2(U_1)$},
	xtick={0,250},
	ytick={0,0.5,1,1.5,2},
	y tick label style={/pgf/number format/1000 sep=},
	]
	\addplot[blue,semithick] file {wykresy/z3Y2U1.txt};
	\nextgroupplot
	[
	xmin=0,xmax=250,ymin=0,ymax=2,
	xlabel={$k$},
	ylabel={$Y_2(U_2)$},
	xtick={0,250},
	ytick={0,0.5,1,1.5,2},
	y tick label style={/pgf/number format/1000 sep=},
	]
	\addplot[green,semithick] file {wykresy/z3Y2U2.txt};
	\end{groupplot}
	\end{tikzpicture}
	\caption{Znormalizowane odpowiedzi skokowe}
	\label{normal}
\end{figure}