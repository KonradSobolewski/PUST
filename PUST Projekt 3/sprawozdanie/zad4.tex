\chapter{Zadanie 4: Algorytmy PID i DMC}
\section{Cyfrowy algorytm PID}

W projekcie zosta� wykorzystany regulator cyfrowy $PID$, kt�rego parametry s� opisane poni�szymi wzorami, gdzie 
$K$ - wzmocnienie cz�onu P , $T_p$ - czas pr�bkowania, $T_i$ - czas zdwojenia cz�onu ca�kuj�cego $I$, $T_d$ - czas wyprzedzenia cz�onu r�niczkuj�cego $D$ , $n_u$ - ilo�� sterowa� , $n_y$ - ilo�� wyj��. 

\begin{equation}
{r_0}^j=K^j*(1+T_p/(2*{T_i}^j)+{T_d}^j/T_p) \quad \forall j \in <1,n_u>
\label{r0}
\end{equation}


\begin{equation}
{r_1}^j=K^j*(T_p/(2*{T_i}^j)-2*{T_d}^j/T_p-1) \quad \forall j \in <1,n_u> 
\label{r1}
\end{equation}

\begin{equation}
{r_2}^j=K*{T_d}^j/T_p \quad \forall j \in <1,n_u>
\label{r2}
\end{equation}
W ka�dej iteracji p�tli sterowania s� obliczane uchyby wyj�� obiektu.
\begin{equation}
e(k)^j = Y^{\mathrm{zad}}(k)^j - Y(k)^j \quad \forall j \in <1,n_y>
\label{uchyb}
\end{equation}
Sterowania regulatora zostaj� wyliczone na bie��c� chwile przy u�yciu wzoru:
\begin{equation}
U(k)^j = {r_2}^j*e(k-2)^i + {r_1}^j*e(k-1)^i + {r_0}^j*e(k)^i + U(k-1)^j \quad gdzie \quad j \in <1,n_u>, \quad i \in <1,n_y>
\label{Uk}
\end{equation}

\section{Analityczny algorytm DMC}
Do oblicze� wykorzystujemy nast�puj�ce wzory:
\begin{equation}
\boldsymbol{y}^{\mathrm{zad}}(k)=\left[
\begin{array}{c}
y^{\mathrm{zad}}_1(k)\\
\vdots\\
y^{\mathrm{zad}}_{ny}(k)
\end{array}
\right]_{\mathrm{n_yx1}}
\label{yzadm}
\end{equation}

\begin{equation}
\boldsymbol{y}(k)=\left[
\begin{array}{c}
y_1(k)\\
\vdots\\
y_{ny}(k)
\end{array}
\right]_{\mathrm{n_yx1}}
\label{ym}
\end{equation}

\begin{equation}
\boldsymbol{u}(k)=\left[
\begin{array}{c}
 u_1(k)\\
\vdots\\
u_{n_u}(k)
\end{array}
\right]_{\mathrm{n_ux1}}
\label{Um}
\end{equation}

\begin{equation}
\triangle\boldsymbol{u}(k)=\left[
\begin{array}{c}
\triangle u_1(k)\\
\vdots\\
\triangle u_{n_u}(k)
\end{array}
\right]_{\mathrm{n_ux1}}
\label{dUm}
\end{equation}

\begin{equation}
\boldsymbol{Y}^{\mathrm{zad}}(k)=\left[
\begin{array}{c}
y^{\mathrm{zad}}(k|k)\\
\vdots\\
y^{\mathrm{zad}}(k|k)
\end{array}
\right]_{\mathrm{N*n_yx1}}
\label{yzad}
\end{equation}

\begin{equation}
\triangle\boldsymbol{U}(k)=\left[
\begin{array}{c}
\triangle u_1(k|k)\\
\vdots\\
\triangle u_{n_u}(k+N_u-1|k)
\end{array}
\right]_{\mathrm{N*n_yx1}}
\label{dUms}
\end{equation}

\begin{equation}
\triangle\boldsymbol{U^P}(k)=\left[
\begin{array}{c}
\triangle u(k-1)\\
\vdots\\
\triangle u(k-(D-1))
\end{array}
\right]_{\mathrm{(D-1)*n_ux1}}
\label{dUPm}
\end{equation}

\begin{equation}
\boldsymbol{S_l}=\left[
\begin{array}
{cccc}
s^{11}_l & s^{12}_l & \ldots & s^{1n_u}_l\\
s^{21}_l & s^{22}_l & \ldots & s^{2n_u}_l\\
\vdots & \vdots & \ddots & \vdots\\
s^{n_y1}_l& s^{n_y2}_l & \ldots &  s^{n_yn_u}_l
\end{array}
\right]_{\mathrm{n_yxn_u}}
\label{S}
, l=1,...,D.
\end{equation}

\begin{equation}
\boldsymbol{M}=\left[
\begin{array}
{cccc}
S_{1} & 0 & \ldots & 0\\
S_{2} & S_{1} & \ldots & 0\\
\vdots & \vdots & \ddots & \vdots\\
S_{N} & S_{N-1} & \ldots &  S_{N-N_{\mathrm{u}}+1}
\end{array}
\right]_{\mathrm{(N*n_y)x(N_u*n_u)}}
\label{Mm}
\end{equation}

\begin{equation}
\boldsymbol{M^P}=\left[
\begin{array}
{cccc}
S_{2}-S_{1} & S_{3}-S_{2} & \ldots & S_{D}-S_{D-1}\\
S_{3}-S_{1} & S_{4}-S_{2} & \ldots & S_{D+1}-S_{D-1}\\
\vdots & \vdots & \ddots & \vdots\\
S_{N+1}-S_{1} & S_{N+2}-S_{2} & \ldots &  S_{N+D-1}-S_{D-1}
\end{array}
\right]_{\mathrm{(N*n_y)x((D-1)*n_u)}}
\label{MPm}
\end{equation}

\begin{equation}
Y^0(k)=Y(k)+M^P
\triangle U^P(k)
\label{Y0}
\end{equation}

\begin{equation}
K=(M^TM+\lambda*I)^{-1}M^T
\label{K}
\end{equation}

\begin{equation}
\triangle U(k)=K(Y^{zad}(k)-Y^0(k))
\label{dU1}
\end{equation}

W naszej regulacji potrzebujemy wyznaczy� tylko pierwszy element macierzy $\triangle U(k)$ czyli $\triangle u(k|k)$. W tym celu rozwijamy wz�r do postaci:

\begin{equation}
\triangle u(k|k)=k_ee(k)-k_u\triangle\boldsymbol U^P
\label{dukk}
\end{equation}

gdzie:

\begin{equation}
e(k)=y^{zad}(k)-y(k)
\label{e}
\end{equation}

Poniewa� nasze $n_u=2$ i $n_y=2$ to:

\begin{equation}
\boldsymbol{k_e}=\left[
\begin{array}
{cccc}
k_e^1 & k_e^2 \\
k_e^3 & k_e^4 \\
\end{array}\right]
\label{ke}
\end{equation}

Dla nieparzystych j $k_e^j$ to suma nieparzystych element�w (j+1)/2-tego wiersza macierzy K. Dla parzystych j $k_e^j$ to suma parzystych element�w j/2-tego wiersza macierzy K.

\begin{equation}
k_u=kM^P
\label{ku}
\end{equation}

k to oznaczenie macierzy b�d�cej $n_u$ pocz�tkowymi wierszami macierzy K (u nas 2 pierwsze wiersze). Aktualne sterowanie otrzymujemy poprzez zsumowanie poprzedniego sterowania i aktualnie wyliczonego $\triangle u(k|k)$. 