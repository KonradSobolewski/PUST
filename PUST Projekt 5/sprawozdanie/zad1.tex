\chapter{Zadanie 1: Weryfikacja punktu pracy}
Pierwszym poleceniem by�o zweryfikowanie poprawno�ci punktu pracy obiektu. Uda�o si� to osi�gn�� za pomoc� prostego sprawdzenia, przy jakich warto�ciach wyj�� stabilizuje si� obiekt przy sta�ym sterowaniu, r�wnym jego warto�ci w punkcie pracy ($U1_{pp}=0$,$U2_{pp}=0$,$U3_{pp}=0$,$U4_{pp}=0$). Eksperyment potwierdzi� wcze�niej podane warto�ci wyj�� ($Y1_{pp}=0$,$Y2_{pp}=0$,$Y3_{pp}=0$), a jego przebieg obrazuje wykres \ref{fig:punktpracy}.

\begin{figure}[tb]
	\centering
	\begin{tikzpicture}
	\begin{groupplot}[group style={group size=1 by 3,vertical sep={2 cm}},
	width=0.9\textwidth,height=0.3\textwidth]
	\nextgroupplot
	[
	xmin=0,xmax=100,ymin=-1,ymax=1,
	xlabel={$k$},
	ylabel={$y1$},
	xtick={0,50,100},
	ytick={-1,-0.5,0,0.5,1},
	y tick label style={/pgf/number format/1000 sep=},
	]
	\addplot[blue,semithick] file {wykresy/z1/z1Y1.txt};
	\nextgroupplot
	[
	xmin=0,xmax=100,ymin=-1,ymax=1,
	xlabel={$k$},
	ylabel={$y2$},
	xtick={0,50,100},
	ytick={-1,-0.5,0,0.5,1},
	y tick label style={/pgf/number format/1000 sep=},
	]
	\addplot[blue,semithick] file {wykresy/z1/z1Y2.txt};
	\nextgroupplot
	[
	xmin=0,xmax=100,ymin=-1,ymax=1,
	xlabel={$k$},
	ylabel={$y3$},
	xtick={0,50,100},
	ytick={-1,-0.5,0,0.5,1},
	y tick label style={/pgf/number format/1000 sep=},
	]
	\addplot[blue,semithick] file {wykresy/z1/z1Y3.txt};
	\end{groupplot}
	\end{tikzpicture}
	\caption{Zachowanie obiektu w punkcie pracy}
	\label{fig:punktpracy}
\end{figure}