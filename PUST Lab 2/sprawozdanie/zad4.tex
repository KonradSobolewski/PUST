\chapter{Zadanie 4: Algorytmy PID i DMC}
\section{PID}
Kod programu zawieraj�cy model obiektu stanowiska oraz ograniczenia $$0 <= U <= 100$$ zosta� zawarty w folderze sprawozdania. 
Do wymodelowania obiektu pos�u�yli�my si� funkcj� $fmincon$ , kt�ra zwr�ci�a nam parametry obiektu $$T1 = 91,607279510963700$$ $$T2 = 6,971579071504170$$ 
$$K_p = 0,997200696364098$$ $$TD = 2$$ oraz aproksymuj�cym r�wnaniem r�nicowym 
$$Y(k) = b_1U(k-TD-1) + b_2U(k-TD-2) - a_1Y(k-1) - a_2Y(k-2)$$

Powy�sze parametry wyznaczone dla znormalizowanego skoku jednostkowego udokumentowane w poprzednim punkcie, powodowa�y gorszy przebieg wyj�cia oraz sterowania obiektu, dlatego nowe parametry zosta�y wyznaczone dla skoku jednostkowego z punktu pracy, aby zachowa� ich zbli�one do rzeczywisto�ci trajektorie.

\section{DMC}