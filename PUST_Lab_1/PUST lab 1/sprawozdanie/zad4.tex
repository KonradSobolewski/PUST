\chapter{Zadanie 4: Algorytmy DMC}
\section{Analityczny algorytm DMC}
Do oblicze� wykorzystujemy nast�puj�ce wzory:
\begin{equation}
\boldsymbol{Y}^{\mathrm{zad}}(k)=\left[
\begin{array}{c}
Y^{\mathrm{zad}}(k)\\
\vdots\\
Y^{\mathrm{zad}}(k)
\end{array}
\right]_{\mathrm{Nx1}}
\label{yzadm}
\end{equation}

\begin{equation}
\boldsymbol{Y}(k)=\left[
\begin{array}{c}
y(k)\\
\vdots\\
y(k)
\end{array}
\right]_{\mathrm{Nx1}}
\label{ym}
\end{equation}

\begin{equation}
\triangle\boldsymbol{U}(k)=\left[
\begin{array}{c}
\triangle u(k|k)\\
\vdots\\
\triangle u(k+N_u -1 |k)
\end{array}
\right]_{\mathrm{N_ux1}}
\label{dUm}
\end{equation}

\begin{equation}
\triangle\boldsymbol{U^P}(k)=\left[
\begin{array}{c}
\triangle u(k-1)\\
\vdots\\
\triangle u(k-(D-1))
\end{array}
\right]_{\mathrm{(D-1)x1}}
\label{dUPm}
\end{equation}

\begin{equation}
\boldsymbol{M}=\left[
\begin{array}
{cccc}
s_{1} & 0 & \ldots & 0\\
s_{2} & s_{1} & \ldots & 0\\
\vdots & \vdots & \ddots & \vdots\\
s_{N} & s_{N-1} & \ldots &  s_{N-N_{\mathrm{u}}+1}
\end{array}
\right]_{\mathrm{NxN_u}}
\label{Mm}
\end{equation}

\begin{equation}
\boldsymbol{M^P}=\left[
\begin{array}
{cccc}
s_{2}-s_{1} & s_{3}-s_{2} & \ldots & s_{D}-s_{D-1}\\
s_{3}-s_{1} & s_{4}-s_{2} & \ldots & s_{D+1}-s_{D-1}\\
\vdots & \vdots & \ddots & \vdots\\
s_{N+1}-s_{1} & s_{N+2}-s_{2} & \ldots &  s_{N+D-1}-S_{D-1}
\end{array}
\right]_{\mathrm{NxD-1}}
\label{MPm}
\end{equation}

\begin{equation}
Y^0(k)=Y(k)+M^P
\triangle U^P(k)
\label{Y0}
\end{equation}

\begin{equation}
K=(M^TM+\lambda*I)^{-1}M^T
\label{K}
\end{equation}

\begin{equation}
\triangle U(k)=K(Y^{zad}(k)-Y^0(k))
\label{dU1}
\end{equation}

W naszej regulacji potrzebujemy wyznaczy� tylko pierwszy element macierzy $\triangle U(k)$ czyli $\triangle u(k|k)$. W tym celu rozwijamy wz�r do postaci:

\begin{equation}
\triangle u(k|k)=k_ee(k)-k_u\triangle\boldsymbol U^P
\label{dukk}
\end{equation}

gdzie:

\begin{equation}
e(k)=Y^{zad}(k)-Y(k)
\label{e}
\end{equation}

\begin{equation}
k_e=\sum_{i=1}^N K(1,i)
\label{ke}
\end{equation}

\begin{equation}
k_u=kM^P
\label{ku}
\end{equation}

k to oznaczenie pierwszego wiersza macierzy K. Aktualne sterowanie otrzymujemy poprzez zsumowanie poprzedniego sterowania i aktualnie wyliczonego $\triangle u(k|k)$. 