\chapter{Zadanie 4: Algorytmy PID i DMC}
\section{PID}
Kod programu zawieraj?cy model obiektu stanowiska oraz ograniczenia $$0 <= U <= 100$$ zosta?y zawarty w folderze sprawozdania. 
Do wymodelowania obiektu pos?u?yli?my si? funkcj? $fmincon$ , kt�ra zwr�ci?a nam parametry obiektu $$T1 = 91.607279510963700$$ $$T2 = 6.971579071504170$$ 
$$K_p = 0.997200696364098$$ $$TD = 2$$ oraz aproksymuj?cym r�wnaniem r�?nicowym 
$$Y(k) = b1*U(k-Td-1) + b2*U(k-Td-2) - a1*Y(k-1) - a2*Y(k-2)$$

Powy?sze parametry wyznaczone dla znormalizowanego skoku jednostkowego udokomuntowane w poprzednim punkcie, powodowa?y gorszy przebieg wyj?cia oraz sterowania obiektu, dlatego nowe parametry zosta?y wyznaczone dla skoku jednostkowego z punktu pracy, aby zachowa? ich zbli?on? do rzeczywisto?ci trajektorie.

\section{DMC}