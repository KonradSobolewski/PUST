\chapter{Zadanie 3: Znormalizowana odpowied� skokowa}
\label{sec:zad3}
ffffNormalizacja odpowiedzi skokowej polega na przesuni�ciu ka�dej warto�ci wyj�cia obiektu o warto�� w punkcie pracy, a nast�pnie
podzielenie jej przez d�ugo�� skoku sterowania.
\begin{equation}
S=(Y-Ypp)/dU;
\end{equation}
 W ten spos�b otrzymujemy warto�� odpowiedzi skokowej w formie w jakiej otrzymaliby�my j� robi�c skok jednostkowy. Taka odpowied� skokowa jest gotowa do u�ytku w regulatorze $DMC$. Wyniki zosta�y przedstawione na wykresie \ref{fig:z3}.
\begin{figure}[tb]
\centering
\begin{tikzpicture}
\begin{axis}[
width=0.9\textwidth,
xmin=0,xmax=235,ymin=0,ymax=2,
xlabel={$k$},
ylabel={$s$},
xtick={0,50,100,150,200,250,300},
ytick={0,0.5,1,1.5,2},
y tick label style={/pgf/number format/1000 sep=},
]
\addplot[blue,semithick] file {wykresy/z3.txt};
\end{axis}
\end{tikzpicture}
\caption{Wykres znormalizowanej odpowiedzi skokowej}
\label{fig:z3}
\end{figure}