\chapter{Zadanie 2: Mechanizm zabezpieczaj�cy}

Nast�pnym zadaniem do wykonanie by�a implementacja mechanizmu zabezpieczaj�cego przed uszkodzeniem grza�ek. Polega to na implementacji mechanizmu symuluj�cego przekroczenie temperatury 150 stopni celsjusza.
Z powodu braku mo�liwo�ci otworzenia wspomnianego pliku poza laboratorium, implementacja jest zawarta wy��cznie w pliku o nast�puj�cej �cie�ce:
$$stanowiskogrzewcze.gx3 > FIXED SCAN > PID $$
Na rys . \ref{overheat} znajduj�ce si� wykresy zebrane przez �rodowisko $Matlab$ przedstawiaj�ce poprawno�� implementacji naszego zabezpieczenia. 

\begin{figure}[tb]
	\centering
	\begin{tikzpicture}
	\begin{groupplot}[group style={group size=1 by 2,vertical sep={2 cm}},
	width=0.9\textwidth,height=0.4\textwidth]
	\nextgroupplot
	[
	xmin=0,xmax=35,ymin=-20,ymax=450,
	xlabel={$k$},
	ylabel={$U$},
	xtick={0,5,10,15,20,25,30,35},
	ytick={0,50,100,150,200,250,300,350,400,450},
	y tick label style={/pgf/number format/1000 sep=},
	]
	\addplot[blue,semithick] file {wykresy/zad2/temp1U1.txt};
	\addplot[green,semithick] file {wykresy/zad2/temp1U2.txt};
	\nextgroupplot
	[
	xmin=0,xmax=35,ymin=0,ymax=35000,
	xlabel={$k$},
	ylabel={$Y$},
	xtick={0,5,10,15,20,25,30,35},
	ytick={0,5000,10000,15000,20000,25000,30000,35000},
	y tick label style={/pgf/number format/1000 sep=},
	legend pos=north east,
	]
	\addplot[blue,semithick] file {wykresy/zad2/temp1Y1.txt};
	\addplot[green,semithick] file {wykresy/zad2/temp1Y2.txt};
	\legend{$Y1(k)$,$Y2(k)$}
	\end{groupplot}
	\end{tikzpicture}
	\caption{Mechanizm zabezpieczaj�cy}
	\label{overheat}
\end{figure}
\FloatBarrier 
