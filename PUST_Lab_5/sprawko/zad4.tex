\chapter{Zadanie 4: DMC}

Na tym etapie przyst�pili�my do implementacji algorytmu $DMC$ z ograniczeniami na stanowisku grzej�co - ch�odz�cym. Z powodu braku mo�liwo�ci otworzenia wspomnianego pliku poza laboratorium, implementacja jest zawarta wy��cznie w pliku o nast�puj�cej �cie�ce:
$$stanowiskogrzewcze.gx3 > FIXED SCAN > DMC $$

\section{Odpowiedzi skokowe}

Pami�taj�c, �e obiekt jest symetryczny postanowili�my zebra� odpowiedzi skokowe wy��cznie dla zmiany sterowania pierwszej grza�ki. Zebrane oraz zaproksymowane wyniki przedstawione s� na rys. \ref{odp}

\begin{figure}[tb]
	\centering
	\begin{tikzpicture}
	\begin{groupplot}[group style={group size=1 by 4,vertical sep={2 cm}},
	width=0.9\textwidth,height=0.25\textwidth]
	\nextgroupplot
	[
	xmin=0,xmax=100,ymin=3800,ymax=4300,
	xlabel={$k$},
	ylabel={$Y1(U1)$},
	xtick={0,100},
	ytick={3800,4300},
	y tick label style={/pgf/number format/1000 sep=},
	legend pos=south east,
	]
	\addplot[blue,semithick] file {wykresy/zad4skok/skokU1Y1.txt};
	\legend{$Y1(k)$}
	\nextgroupplot
	[
	xmin=0,xmax=100,ymin=3800,ymax=4300,
	xlabel={$k$},
	ylabel={$Y2(U1)$},
	xtick={0,100},
	ytick={3800,4300},
	y tick label style={/pgf/number format/1000 sep=},
	legend pos=south east,
	]
	\addplot[blue,semithick] file {wykresy/zad4skok/skokU1Y2.txt};
	\legend{$Y2(k)$}
	\nextgroupplot
	[
	xmin=0,xmax=100,ymin=0,ymax=0.28,
	xlabel={$k$},
	ylabel={$Y1(U1)$},
	xtick={0,100},
	ytick={0,0.28},
	y tick label style={/pgf/number format/1000 sep=},
	legend pos=south east,
	]
	\addplot[blue,semithick] file {wykresy/zad4skok/aprskoky1.txt};
	\legend{$aproksymacja_Y1$}
	\nextgroupplot
	[
	xmin=0,xmax=100,ymin=0,ymax=0.28,
	xlabel={$k$},
	ylabel={$Y2(U1)$},
	xtick={0,100},
	ytick={0,0.28},
	y tick label style={/pgf/number format/1000 sep=},
	legend pos=south east,
	]                                                           
	\addplot[blue,semithick] file {wykresy/zad4skok/aprskoky2.txt};
	\legend{$aproksymacja_Y2$}
	\end{groupplot}
	\end{tikzpicture}
	\caption{Odpowiedzi skokowe oraz ich normalizacja}
	\label{odp}
\end{figure}
\FloatBarrier

\section{DMC}

Na rys. \ref{DMC} zosta�y zaprezentowane wyniki dla tor�w sterowania oraz wyj�cia regulatora $DMC$. W trakcie trwania laboratorium uda�o si� nam zebra� przebiegi dla nastaw $N=100,N_u=100,\lambda=1$. Prowadz�cy laboratoria oceni� nasz wygl�d regulacji na zadawalaj�cy oraz zasugerowa� nam porzucenie dalszych bada� i skupienie si� na nast�pnych punktach laboratorium.

\begin{figure}[tb]
	\centering
	\begin{tikzpicture}
	\begin{groupplot}[group style={group size=1 by 4,vertical sep={2 cm}},
	width=0.9\textwidth,height=0.25\textwidth]
	\nextgroupplot
	[
	xmin=0,xmax=224,ymin=200,ymax=700,
	xlabel={$k$},
	ylabel={$U1$},
	xtick={0,224},
	ytick={200,700},
	y tick label style={/pgf/number format/1000 sep=},
	]
	\addplot[blue,semithick] file {wykresy/zad4DMC/dmcU1.txt};
	\nextgroupplot
	[
	xmin=0,xmax=224,ymin=200,ymax=700,
	xlabel={$k$},
	ylabel={$U2$},
	xtick={0,224},
	ytick={200,700},
	y tick label style={/pgf/number format/1000 sep=},
	]
	\addplot[blue,semithick] file {wykresy/zad4DMC/dmcU2.txt};
	\nextgroupplot
	[
	xmin=0,xmax=224,ymin=3800,ymax=4600,
	xlabel={$k$},
	ylabel={$Y1$},
	xtick={0,224},
	ytick={3800,4600},
	y tick label style={/pgf/number format/1000 sep=},
	legend pos=south east,
	]
	\addplot[blue,semithick] file {wykresy/zad4DMC/dmcY1.txt};
	\addplot[green,semithick] file {wykresy/zad4DMC/dmcYzad1.txt};
	\legend{$Y1(k)$,$Yzad1(k)$}
	\nextgroupplot
	[
	xmin=0,xmax=224,ymin=3800,ymax=4700,
	xlabel={$k$},
	ylabel={$Y2$},
	xtick={0,224},
	ytick={3800,4700},
	y tick label style={/pgf/number format/1000 sep=},
	legend pos=south east,
	]
	\addplot[blue,semithick] file {wykresy/zad4DMC/dmcY2.txt};
	\addplot[green,semithick] file {wykresy/zad4DMC/dmcYzad2.txt};
	\legend{$Y2(k)$,$Yzad2(k)$}
	\end{groupplot}
	\end{tikzpicture}
	\caption{DMC $N=100,N_u=100,\lambda=1$}
	\label{DMC}
\end{figure}
\FloatBarrier  