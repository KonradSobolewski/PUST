\chapter{Zadanie 1: Punkt pracy}
Pierwszym poleceniem by�o okre�lenie warto�ci wyj�� obiektu $Y_{pp}$ (pomiaru $T1$ oraz $T3$) w punkcie pracy $U_{pp_1} = 36$ $U_{pp_2} = 41$. Osi�gn�li�my j� ustawiaj�c warto�� sterowania (moc grzania grza�ek $G1$,$G2$) na $U_{pp_1}$ $U_{pp_2}$ i odczekuj�c znaczn� ilo�� czasu (powy�ej 6 minut). Jak wida� na wykresie \ref{skok}, wyj�cia ustabilizowa�y si� w okolicy $Y_{pp_1} = 39$ oraz $Y_{pp_1} = 41$ stopni Celcjusza. 


\begin{figure}[tb]
	\centering
	\begin{tikzpicture}
	\begin{axis}[
	width=0.9\textwidth,
	xmin=5,xmax=90,ymin=3500,ymax=4200,
	xlabel={$k$},
	ylabel={$Y(k)$},
	xtick={5,15,25,35,45,55,65,75,85,90},
	ytick={3500,3600,3700,3800,3900,4000,4100,4200,4300},
	y tick label style={/pgf/number format/1000 sep=},
	legend pos=south east
	]
	\addplot[blue,semithick] file {wykresy/zad1/tempY1.txt};
	\addplot[orange,semithick] file {wykresy/zad1/tempY2.txt};
	
	\legend{$Y_{1pp}$,$Y_{2pp}$}
	\end{axis}
	\end{tikzpicture}
	\caption{Wykresy Y(k) w punktach pracy $U_{pp}$}
	\label{skok}
\end{figure}