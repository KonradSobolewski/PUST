\chapter{Laboratorium: Zadanie 8: Dob�r lambda dla rozmytego regulatora DMC}

\section{Korekta nastaw}
Wykres \ref{rozmytydmc2}, przedstawiajacy dzia�anie regulatora rozmytego ze wsp�czynnikeim $\lambda = 0.5$ nie pozostawia z�udze� - sterowanie jest zbyt gwa�towne i skutkuje oscylacyjnym charakterem warto�ci sterowanej. Aby wyeliminowa� to zjawisko postanowili�my zwi�kszy� lambd�. Przyj�li�my warto�ci $\lambda = 5$ dla pierwszego regulatora, dzia�aj�cego w dolnym zakresie temperatur oraz $\lambda = 2$ dla regulatora z g�rnego przedzia�u. Efekt wida� na poni�szym wykresie \ref{rozmytydmc3}:

\begin{figure}[tb]
\centering
\begin{tikzpicture}
\begin{groupplot}[group style={group size=1 by 2,vertical sep={2 cm}},
width=0.9\textwidth,height=0.5\textwidth]
\nextgroupplot
[
xmin=0,xmax=1210,ymin=-5,ymax=105,
xlabel={$k$},
ylabel={$U(k)$},
xtick={0,100,200,300,400,500,600,700,800,900,1000,1100,1200},
ytick={0,10,20,30,40,50,60,70,80,90,100},
y tick label style={/pgf/number format/1000 sep=},
legend pos=south east,
]
\addplot[blue,semithick] file {wykresy/lab6dmcU3.txt};
\nextgroupplot
[
xmin=0,xmax=1210,ymin=36,ymax=47,
xlabel={$k$},
ylabel={$Y(k)$},
xtick={0,100,200,300,400,500,600,700,800,900,1000,1100,1200},
ytick={36,37,38,39,40,41,42,43,44,45,46},
y tick label style={/pgf/number format/1000 sep=},
legend pos=north east,
]
\addplot[red,semithick] file{wykresy/lab6dmcY3.txt}; 
\addplot[orange,semithick,densely dashed] file{wykresy/lab6Yzad.txt}; 
\legend{$Y$,$Y^{zad}$,$Y_2$,}
\end{groupplot}
\end{tikzpicture}
\caption{Dzia�anie rozmytego regulatora DMC z dwoma lokalnymi regulatorami PID o nastawach: $D^1 = 300, Nu^1 = 150, \lambda^1 = 5$ oraz $D^2 = 300, Nu^2 = 150, \lambda^2 = 2$}
\label{rozmytydmc3}
\end{figure}
\FloatBarrier

\section{Wnioski}

Jak wida�, jako�� regulacji poprawi�a si�, chocia� wci�� jest daleka od optymalnej. Niestety z powodu braku czasu nie byli�my w stanie przetestowa� innych nastaw. Je�li pozwoli�by nam na to czas, przeprowadziliby�my kolejne eksperymenty z jeszcze wi�kszymi wska�nikami $\lambda$.