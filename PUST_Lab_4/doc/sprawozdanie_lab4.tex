\documentclass[a4paper,titlepage,11pt,twosides,floatssmall]{mwrep}
\usepackage[left=2.5cm,right=2.5cm,top=2.5cm,bottom=2.5cm]{geometry}
\usepackage[OT1]{fontenc}
\usepackage{polski}
\usepackage{amsmath}
\usepackage{amsfonts}
\usepackage{amssymb}
\usepackage{graphicx}
\usepackage{url}
\usepackage{tikz}
\usetikzlibrary{arrows,calc,decorations.markings,math,arrows.meta}
\usepackage{rotating}
\usepackage[percent]{overpic}
\usepackage[cp1250]{inputenc}
\usepackage{xcolor}
\usepackage{pgfplots}
\usetikzlibrary{pgfplots.groupplots}
\usepackage{listings}
\usepackage{matlab-prettifier}
\usepackage{siunitx}
\usepackage[section]{placeins}
\definecolor{szary}{rgb}{0.95,0.95,0.95}
\sisetup{detect-weight,exponent-product=\cdot,output-decimal-marker={,},per-mode=symbol,binary-units=true,range-phrase={-},range-units=single}
\SendSettingsToPgf

%konfiguracje pakietu listings
\lstset{
	backgroundcolor=\color{szary},
	frame=single,
	breaklines=true,
}
\lstdefinestyle{customlatex}{
	basicstyle=\footnotesize\ttfamily,
	%basicstyle=\small\ttfamily,
}
\lstdefinestyle{customc}{
	breaklines=true,
	frame=tb,
	language=C,
	xleftmargin=0pt,
	showstringspaces=false,
	basicstyle=\small\ttfamily,
	keywordstyle=\bfseries\color{green!40!black},
	commentstyle=\itshape\color{purple!40!black},
	identifierstyle=\color{blue},
	stringstyle=\color{orange},
}
\lstdefinestyle{custommatlab}{
	captionpos=t,
	breaklines=true,
	frame=tb,
	xleftmargin=0pt,
	language=matlab,
	showstringspaces=false,
	%basicstyle=\footnotesize\ttfamily,
	basicstyle=\scriptsize\ttfamily,
	keywordstyle=\bfseries\color{green!40!black},
	commentstyle=\itshape\color{purple!40!black},
	identifierstyle=\color{blue},
	stringstyle=\color{orange},
}

%wymiar tekstu (bez �ywej paginy)
\textwidth 160mm \textheight 247mm

%ustawienia pakietu pgfplots
\pgfplotsset{
tick label style={font=\scriptsize},
label style={font=\small},
legend style={font=\small},
title style={font=\small}
}

\def\figurename{Rys.}
\def\tablename{Tab.}

%konfiguracja liczby p�ywaj�cych element�w
\setcounter{topnumber}{0}%2
\setcounter{bottomnumber}{3}%1
\setcounter{totalnumber}{5}%3
\renewcommand{\textfraction}{0.01}%0.2
\renewcommand{\topfraction}{0.95}%0.7
\renewcommand{\bottomfraction}{0.95}%0.3
\renewcommand{\floatpagefraction}{0.35}%0.5

\begin{document}
\frenchspacing
\pagestyle{uheadings}

%strona tytu�owa
\title{\bf Sprawozdanie z projektu nr 4 oraz laboratorium nr 4\vskip 0.1cm}
\author{Sobolewski Konrad, R�a�ski Antoni, Gie�dowski Daniel }
\date{2017}

\makeatletter
\renewcommand{\maketitle}{\begin{titlepage}
\begin{center}{\LARGE {\bf
Wydzia� Elektroniki i Technik Informacyjnych}}\\
\vspace{0.4cm}
{\LARGE {\bf Politechnika Warszawska}}\\
\vspace{0.3cm}
\end{center}
\vspace{5cm}
\begin{center}
{\bf \LARGE Projektowanie uk�ad�w sterowania\\ (projekt grupowy) \vskip 0.1cm}
\end{center}
\vspace{1cm}
\begin{center}
{\bf \LARGE \@title}
\end{center}
\vspace{2cm}
\begin{center}
{\bf \Large \@author \par}
\end{center}
\vspace*{\stretch{6}}
\begin{center}
\bf{\large{Warszawa, \@date\vskip 0.1cm}}
\end{center}6
\end{titlepage}
}
\makeatother

\maketitle

\tableofcontents
%\chapter{Opis obiektu}
\label{sec:opis}
Obiekt u�ywany w projekcie opisany jest dan� przez prowadz�cego funkcj�:
\begin{equation}
  Y(k)=symulacja\_obiektu3y(U(k-5),U(k-6),Y(k-1),Y(k-2))
\end{equation}
gdzie $k$ jest aktualn� chwil� symulacji sygna�u pr�bkowanego.
Warto�� sygna��w w punkcie pracy ( w stanie ustalonym ) maj� warto�� $u=y=0$. Okres pr�bkowania obiektu wynosi $T_p=0,5s$. 
%\chapter{Zadanie 1: Punkt pracy}
Pierwszym poleceniem by�o okre�lenie warto�ci wyj�cia obiektu ( pomiaru $T1$ ) w punkcie pracy $U_{pp} = 36$.
Osi�gn�li�my j� ustawi�c warto�� sterowania ( moc grzania grza�ki G1 ) na $U_{pp}$ i odczekuj�c znaczn� ilo�� czasu ( powy�ej 5 min. ).
Ostatecznie wyj�cie ustabilizowa�o si� w pobli�u warto�ci $Y_{pp}=36,5$. 

%\chapter{Zadanie 2: Odpowiedzi skokowe}
\section{Skro�nie odpowiedzi skokowe}
Kolejnym zadaniem by�o wyznaczenie trzech skro�nych odpowiedzi skokowych obiektu. Wykonali�my skoki na grza�ce G1 i zmierzyli�my wyj�cie obiektu na czuniku T3 rozpoczynaj�cych si� z punktu pracy. Wyniki tego eksperymentu, razem z lini� pokazuj�c� po�o�enie punktu pracy, znajduj� si� na rysunku \ref{skoky2u1}. Dla przedstawionych odpowiedzi skok sterowania nast�powa� w chwili $k = 10$, co oznacza, �e dopiero od $k = 11$ wykresy przedstawiaj� w�a�ciwe odpowiedzi skokowe. 

\begin{figure}[tb]
\centering
\begin{tikzpicture}
\begin{axis}[
width=0.9\textwidth,
xmin=0,xmax=310,ymin=41,ymax=44,
xlabel={$k$},
ylabel={$S(k)$},
xtick={0,50,100,150,200,250,300,350},
ytick={36,36.5,38,38.5,40,41,42,43,44},
y tick label style={/pgf/number format/1000 sep=},
]
\addplot[blue,semithick] file {wykresy/z2Y241.000041.0000.txt};
\addplot[green,semithick] file {wykresy/z2Y246.000041.0000.txt};
\addplot[magenta,semithick] file {wykresy/z2Y251.000041.0000.txt};

\legend{$dU_1=\num{5}$,$dU_1=\num{10}$,$dU_1=\num{15}$,$Y_{pp}=\num{41.3}$}
\end{axis}
\end{tikzpicture}
\caption{Wykresy $S_2(k)$ dla r�nych skok�w sterowania z $U_{1pp} = 36$ o dU}
\label{skoky2u1}
\end{figure}

\section{W�a�ciwo�ci statyczne obiektu}
Trudno�ci sprawi�o nam okre�lenie, czy w�a�ciwo�ci statyczne posiadanego obiektu s� liniowe. Bior�c dos�ownie ko�cowe warto�ci wyj�cia skrio�nych odpowiedzi skokowych nale�a�oby stwierdzi�, �e nie s�. Nale�y jednak pami�ta� o nast�puj�cych faktach:


\begin{enumerate}
	\item W okolicy obiektu wyst�powa�y nieuchronne zak��cenia, kt�re w znacz�cy spos�b wp�yn�y na odczyty;
	\item Analizowane s� odpowiedzi skro�ne, kt�re maj� stosunkowo ma�e wzmocnienie i szum b�dzie na nie z tego powodu mocniej wp�ywa�;
	\item W trakcie trwania laboratorium punkt pracy przesuwa� si� w g�r�, co r�wie� utrudnia poprawn� interpretacj� wyniku;
	\item Do�wiadczenia z poprzednich laboratori�w na tym obiekcie wskazuj�, �e jest to obiekt liniowy.
\end{enumerate}

Maj�c na uwadze powy�sze fakty, mo�na zauwa�y� w przebiegach pewne prawid�owo�ci. Przesuwaj�c wykresy, tak, aby mia�y punkt pracy na tym samym poziomie, widzimy, �e: przy skoku sterowania o $5\%$ temperatura podnios�a si� o oko�o $0,4$ stopnia, a przy skoku o $15\%$ o oko�o $1,2$ stopnia. Jak mo�na zauwa�y�, wyj�cie wzros�o o oko�o 3 razy wi�ksz� warto�� przy 3 razy wi�kszym skoku, wi�c zachowa�o si� liniowo. Gdyby za�o�y�, �e przy skoku o $10\%$ nast�pi�y wi�ksze zak��cenia ni� przy innych i odczyt jest zbyt du�y, to mo�na by stwierdzi�, �e charakterystyka statyczna obiektu jest w przybli�eniu liniowa, a jej wzmocnienie statyczne toru $y_2(u_1)$ wynosi oko�o $0,4/5 = 0,08$.

Potwierdzaj� to wykresy odpowiesdzi skokowej nieskro�nej - na \ref{skoky1u1} zosta�y przedstawione charakterystyki $y_1(u_1)$ przesuni�te do wsp�lnego punktu pracy:

\begin{figure}[tb]
	\centering
	\begin{tikzpicture}
	\begin{axis}[
	width=0.9\textwidth,
	xmin=0,xmax=310,ymin=38,ymax=44.5,
	xlabel={$k$},
	ylabel={$S(k)$},
	xtick={0,50,100,150,200,250,300,350},
	ytick={36,36.5,38,39,40,41,42,43,44},
	y tick label style={/pgf/number format/1000 sep=},
	]
	\addplot[blue,semithick] file {wykresy/z2presY141.000041.0000.txt};
	\addplot[green,semithick] file {wykresy/z2presY146.000041.0000.txt};
	\addplot[magenta,semithick] file {wykresy/z2presY151.000041.0000.txt};
	
	\legend{$dU_1=\num{5}$,$dU_1=\num{10}$,$dU_1=\num{15}$,$Y_{pp}=\num{41.3}$}
	\end{axis}
	\end{tikzpicture}
	\caption{Przesuni�te do wsp�lnego punktu pracy wykresy $S_1(k)$ dla r�nych skok�w sterowania z $U_{1pp} = 36$ o dU}
	\label{skoky1u1}
\end{figure}


Wida� na nich statytyczny charakter przebieg�w - odleg�o�ci mi�dzy warto�ciami ko�cowymi poszczeg�lnych skok�w s� bliskie tej samej warto�ci. Odczytujemy z niej wzmocnienie statyczne toru $y_1(u_1)$, kt�re wynosi oko�o $2,85/10 = 0,285$. Jak mo�na wnioskowa� z tych warto�ci, wyj�cie $y_1$ na zmian� $u_1$ powinno reagowa� w przybli�eniu 2.5x mocniej, ni� wyj�cie $y_2$.

Poniewa� obiekt jest idealnie symetryczny a obie grza�ki maj� takie same w�a�ciwo�ci, tory  $y_1(u_2)$ i $y_2(u_2)$ b�d� lustrzanym odbiciem tor�w dla wej�cia nr 1. Tak wi�c: wzmocnienie statyczne toru $y_1(u_2) = 0,08$, a wzmocnienie statyczne toru $y_2(u_2) = 0,285$.



\section{W�a�ciwo�ci statyczne procesu $T1(G1,G2)$, $T3(G1,G2)$}

Z powodu ograniczonego czasu na laboratorium, nie byli�my w stanie zebra� odpowiedzi dla r�nych wielu kombinacji obu wej��.
Posiadaj�c jednak warto�ci opisane w poprzednim podpunkcie, mo�emy bez problemu utworzy� tr�jwymiarowe wykresy obrazuj�ce wyj�cie obu wyj�� w zale�no�ci od obu wej��. Tworzymy je w oparciu o dwa fakty: stanowisko jest symetryczne - w�a�ciwo�ci odpowiednich tor�w ($y_1(u_1)$ i $y_2(u_2)$ oraz $y_1(u_2)$ i $y_2(u_1)$) s� identyczne  oraz zachowane s� w�a�ciwo�ci liniowe, mo�emy wi�c skorzysta� z zasady superpozycji: przyk�adowo, wyj�cie w stanie ustalonym $y_1$ dla $du_1 = 10$ i $du_2 = 15$ b�dzie wynikiem sumy: warto�ci $y_1$ w punkcie pracy, r�nicy mi�dzy warto�ci� ko�cow� i pocz�tkow� odpowiedzi skokowej $y_1(u_1)$ dla skoku $du_1 = 10$ oraz r�nicy mi�dzy warto�ci� ko�cow� i pocz�tkow� odpowiedzi skokowej $y_2(u_1)$ dla skoku $du_1 = 15$.

Kompletne charakterystyki przedstawiaj� poni�sze wykresy \ref{y1u1u2} i \ref{y2u1u2}.

\begin{figure}[tb]
	\centering
	\begin{tikzpicture}
	\begin{axis}[
	width=0.7\textwidth,
	zmin=0,zmax=6,
	xlabel={$dU_1$},
	ylabel={$dU_2$},
	zlabel={$dY_1$},
	xtick={0,5,10,15},
	ytick={0,5,10,15},
	ztick={0,2,4,6,8},
	%ytick={19,20,21},
	%y tick label style={/pgf/number format/1000 sep=},
	]
	\addplot3[mesh, mesh/cols=4] file {wykresy/z2Y1U1U2.txt};
	\end{axis}
	\end{tikzpicture}
	\caption{$y_1(u_1,u_2)$}
	\label{y1u1u2}
\end{figure}
\FloatBarrier

\begin{figure}[tb]
	\centering
	\begin{tikzpicture}
	\begin{axis}[
	width=0.7\textwidth,
	zmin=0,zmax=6,
	xlabel={$dU_1$},
	ylabel={$dU_2$},
	zlabel={$dY_2$},
	xtick={0,5,10,15},
	ytick={0,5,10,15},
	ztick={0,2,4,6,8},
	%ytick={19,20,21},
	%y tick label style={/pgf/number format/1000 sep=},
	]
	\addplot3[mesh, mesh/cols=4] file {wykresy/z2Y2U1U2.txt};
	\end{axis}
	\end{tikzpicture}
	\caption{$y_2(u_1,u_2)$}
	\label{y2u1u2}
\end{figure}
\FloatBarrier

Nale�y zaznaczy�, �e nie tworz� one idealnych p�aszczyzn - wynika to z istoty rzeczywistego obiektu, jakim jest stanowisko grzewcze oraz zak��ce� otoczenia, kt�rych obecno�� powoduje obarczenie odczyt�w nieuchronnym b��dem.




%\chapter{Zadanie 3: Znormalizowana odpowied� skokowa}
Kolejnym poleceniem by�o wyznaczy� znormalizowan� odpowied� skokow� (tak� jaka wymagana jest do algorytmu DMC) i zaproksymowa� j�, u�ywaj�c w tym celu cz�onu inercyjnego drugiego rz�du z op�nieniem. Cz�on posiada 4 parametry: $T_1$, $T_2$, $K$ (dalej oznaczane jako $K_p$) i $T_d$ (w dalszej cz�ci sprawozdania oznaczane jako $TD$). Cz�on jest opisany wzorami powsta�ymi po przekszta�ceniu jego transmitancji:
\begin{equation}
\alpha_1=e^{-\frac{1}{T_1}}
\end{equation}
\begin{equation}
\alpha_2=e^{-\frac{1}{T_2}}
\end{equation}
\begin{equation}
a_1=-\alpha_1-\alpha_2
\end{equation}
\begin{equation}
a_1=\alpha_1\alpha_2
\end{equation}
\begin{equation}
b_1=\frac{K_p}{T_1-T_2}[T_1(1-\alpha_1)-T_2(1-\alpha_2)]
\end{equation}
\begin{equation}
b_1=\frac{K_p}{T_1-T_2}[\alpha_1T_2(1-\alpha_2)-\alpha_2T_1(1-\alpha_1)]
\end{equation}
\begin{equation}
y(k)=b_1u(k-TD-1)+b_2u(k-TD-2)-a_1y(k-1)-a_2y(k-2)
\end{equation}
\\
W celu doboru parametr�w cz�onu wykorzystano funkcj� $fmincon$. Jako pocz�tkowe warto�ci dobieranych parametr�w wybrali�my $[11,10,1,10]$, 11 i 10 dla $T_1$ i $T_2$ �eby nie by�y takie same, 1 dla $K_p$, bo przy dotychczas zebranych przebiegach nie spodziewali�my si� du�ego wzmocnienia dla tego obiektu i 10 dla $TD$, bo przez tyle czasu przy skoku o 15 warto�ci odpowiedzi skokowej nie wzrastaj� powy�ej 0. Od do�u ograniczyli�my wszystkie parametry zerami. Od g�ry ograniczyli�my je warto�ciami $[1000,1000,20,30]$, tak by ka�dy parametr mia� przedzia� dostosowany do swoich potrzeb (du�e zmiany dla $T_1$ i $T_2$, ma�e zmiany dla $K_p$, $TD$ s�dz�c po wykresach nie powinno przekroczy� 30). Jako odpowied� toru sterowanie-wyj�cie do znormalizowania wybrali�my t� dla skoku o 15,a dla toru zak��cenie-wyj�cie t� dla skoku o 30. W wyniku normalizacji przekszta�cili�my je do odpowiedzi jakie mieliby�my po skoku jednostkowym. Nast�pnie po wykonaniu aproksymacji otrzymali�my parametry cz�onu r�wne $T_1$=89,0864, $T_2$=0,00031814, $K_p$=0,39431 i $Td$=14 przy b��dzie optymalizacji $e$=0,040621 dla toru sterowanie-wyj�cie oraz $T_1$=40.0697, $T_2$=30.9178, $K_p$=0.1266 i $Td$=14 przy b��dzie optymalizacji $e$=0,012445 dla toru zak��cenie-wyj�cie. Znormalizowan� odpowiedzi i ich aproksymacj� przedstawili�my na wykresach \ref{norm_skoku} i \ref{norm_skokz}.

\begin{figure}[tb]
\centering
\begin{tikzpicture}
\begin{axis}[
width=0.9\textwidth,
xmin=0,xmax=300,ymin=0,ymax=0.4,
xlabel={$k$},
ylabel={$S(k)$},
xtick={0,50,100,150,200,250,300},
ytick={0,0.1,0.2,0.3,0.4},
y tick label style={/pgf/number format/1000 sep=},
legend pos=south east,
]
\addplot[blue,semithick] file {wykresy/skok15.txt};
\addplot[orange,semithick] file {wykresy/aprskok15.txt};

\legend{odpowied� skokowa,aproksymacja}
\end{axis}
\end{tikzpicture}
\caption{Wykres znormalizowanej odpowiedzi skokowej i jej aproksymacji toru sterowanie-wyj�cie}
\label{norm_skoku}
\end{figure}

\begin{figure}[tb]
	\centering
	\begin{tikzpicture}
	\begin{axis}[
	width=0.9\textwidth,
	xmin=0,xmax=300,ymin=0,ymax=0.14,
	xlabel={$k$},
	ylabel={$S(k)$},
	xtick={0,50,100,150,200,250,300},
	ytick={0,0.02,0.04,0.06,0.08,0.1,0.12,0.14},
	y tick label style={/pgf/number format/1000 sep=},
	legend pos=south east,
	]
	\addplot[blue,semithick] file {wykresy/skokZ.txt};
	\addplot[orange,semithick] file {wykresy/aprskok.txt};
	
	\legend{odpowied� skokowa,aproksymacja}
	\end{axis}
	\end{tikzpicture}
	\caption{Wykres znormalizowanej odpowiedzi skokowej i jej aproksymacji toru zak��cenie-wyj�cie}
	\label{norm_skokz}
\end{figure}
%\chapter{Zadanie 4: Strojenie regulator�w}
Nast�pnym zadaniem by�o wyznaczenie optymalnych parametr�w algorytm�w PID i DMC odpowiednio za pomoc� metody in�ynierskiej(PID) i eksperymentalnej(DMC). Jako�� regulacji oceniana by�a wizualnie - na podstawie wykres�w - oraz obliczeniowo na podstawie wska�nika jako�ci regulacji. Wz�r na ten wska�nik znajduje si� poni�ej.
\begin{equation}
E=\sum_{k=1}^{k_{konc}}(Y^{zad}(k)-Y(k))^2
\end{equation}
\section{PID}
Nastawy PID dobieramy w��czaj�c w tym samym czasie tylko jeden tor regulacji z istniej�cych trzech i dobieraj�c jego parametry. Na ko�cu ��czymy 3 tory i korygujemy nastawy. Stroj�c jeden tor nie b�dziemy si� przejmowa� innymi, wi�c nie umieszczali�my ich wykres�w, a przedstawione warto�ci b��d�w b�d� sum� tylko z tego jednego toru.
\subsection{Konfiguracja}
Poniewa� posiadamy 4 wej�cia i 3 wyj�cia w celu napisania regulatora PID jedno z wej�� b�dziemy musieli odrzuci�. Pozostaje zdecydowa� kt�re. W tym celu tworzymy macierz wzmocnie� KKK zawieraj�c� wzmocnienia statyczne wszystkich tor�w proces�w (wersy odzwierciedlaj� kolejne sterowania, a kolumny kolejne wyj�cia).

\begin{equation}
	KKK=\left[
	\begin{array}
	{ccc}
	0,5 & 1,5 & 1,3\\
	1,25 & 1,1 & 0,4\\
	0,9 & 0,1 & 0,3\\
	1,2 & 0,45 & 1,15
	\end{array}
	\right]
\end{equation}

Oczywi�cie ju� na jej podstawie mogliby�my wybra� niez�� konfiguracj� wyj��, jednak�e my szukamy najlepszej. W tym celu dzielimy macierz KKK na cztery macierze KK$_i$ o wymiarach 3x3 usuwaj�c za ka�dym razem inny wiersz (numer i nowej macierzy to numer usuni�tego wiersza).

\begin{equation}
KK_1=\left[
\begin{array}
{ccc}
1,25 & 1,1 & 0,4\\
0,9 & 0,1 & 0,3\\
1,2 & 0,45 & 1,15
\end{array}
\right]
\end{equation}

\begin{equation}
KK_2=\left[
\begin{array}
{ccc}
0,5 & 1,5 & 1,3\\
0,9 & 0,1 & 0,3\\
1,2 & 0,45 & 1,15
\end{array}
\right]
\end{equation}

\begin{equation}
KK_3=\left[
\begin{array}
{ccc}
0,5 & 1,5 & 1,3\\
1,25 & 1,1 & 0,4\\
1,2 & 0,45 & 1,15
\end{array}
\right]
\end{equation}

\begin{equation}
KK_4=\left[
\begin{array}
{ccc}
0,5 & 1,5 & 1,3\\
1,25 & 1,1 & 0,4\\
0,9 & 0,1 & 0,3
\end{array}
\right]
\end{equation}

Nast�pnie obliczamy w Matlabie wska�niki uwarunkowania tych macierzy, kt�re wynosz� odpowiednio:
 \begin{itemize}
 	\item cond(KK$_1$)=6,7173
 	\item cond(KK$_2$)=11,1599
 	\item cond(KK$_3$)=4,2242
 	\item cond(KK$_4$)=6,9254
 \end{itemize}
Nast�pnie wybieramy t�, kt�rej wska�nik jest najmniejszy (w naszym przypadku KK$_3$) i obliczamy dla niej (w matlabie) macierz KK$_i$.*(KK$_i^{-1}$)'. Nast�pnie wybieramy z obliczonej macierzy 3 elementy, po jednym na ka�dy wiersz i kolumn�, maj�ce warto�ci jak najbli�sze jeden (warto�ci ujemne s� wykluczone). Po�o�enie tych element�w okre�la kt�re sterowanie powinno odpowiada� kt�remu wyj�ciu. Poni�ej przedstawiam wyniki tego r�wnania dla wszystkich maicerzy KK$_i$.

\begin{equation}
Dla \quad KK_1 : \left[
\begin{array}
{ccc}
0,0383 & 1,1362 & -0,1744\\
1,4943 & -0,1465 & -0,3477\\
-0,5325 & 0,0103 & 1,5222
\end{array}
\right]
\end{equation}

\begin{equation}
Dla \quad KK_2 : \left[
\begin{array}
{ccc}
    0,0153&    1,5529&   -0,5683\\
1,5736&    0,1511&   -0,7247\\
-0,5890&   -0,7040 &   2,2929
\end{array}
\right]
\end{equation}

\begin{equation}
Dla \quad KK_3 : \left[
\begin{array}
{ccc}
   -0,2888&    0,7646&    0,5242\\
0,7586&    0,5768&   -0,3354\\
0,5302&   -0,3414&    0,8112
\end{array}
\right]
\end{equation}

\begin{equation}
Dla \quad KK_4 : \left[
\begin{array}
{ccc}
   -0,1447  &  0,0225  &  1,1223\\
0,3992   & 1,1198 &  -0,5190\\
0,7455  & -0,1422&    0,3967
\end{array}
\right]
\end{equation}

Normalnie wybraliby�my jedynie konfiguracj� uzyskan� z macierzy o najni�szym wska�niku uwarunkowania (KK$_3$), jednak�e zadanie nakazuje przetestowa� r�ne konfiguracje regulatora PID. Z tego powodu zdecydowali�my si� wybra� po jednej (najlepszej) konfiguracji z ka�dej z macierzy. B�d� to:
\begin{itemize}
	\item Dla KK1: y1-u3 y2-u2 y3-u4
	\item Dla KK2: y1-u3 y2-u1 y3-u4
	\item Dla KK3: y1-u2 y2-u1 y3-u4
	\item Dla KK4: y1-u3 y2-u2 y3-u1
\end{itemize}
\newpage
\subsection{PID - konfiguracja pierwsza}
Pierwsza konfiguracja naszego PID'a zak�ada, �e wyj�cie pierwsze sterujemy wej�ciem trzecim, wyj�cie drugie wej�ciem drugim, a wyj�cie trzecie wej�ciem czwartym.
\begin{table}
	\centering
	\begin{tabular}{c|c}
		$\boldsymbol{y}$ & $\boldsymbol{u}$\\ \hline
		$y_1$ & $u_3$ \\
		$y_2$ & $u_2$ \\
		$y_3$ & $u_4$ 
		\label{k1}
	\end{tabular}
\caption{Pierwsza konfiguracja}
\end{table}
\subsubsection{Tor pierwszy}
Nastawy PID wyznaczamy metod� in�yniersk�. Oznacza to, �e zaczynamy od wyznaczenia wzmocnienia K. Jego warto�� ustawiamy na po�ow� warto�ci, dla kt�rej obiekt wpada w nieko�cz�ce si� i nierosn�ce oscylacje. Dla pierwszego toru $K_{osc}=13,3642$, co oznacza, �e jako warto�� K przyjmujemy $K=6,6821$. Oscylacje przedstawia poni�szy wykres \ref{k1t1K}.

Nast�pnie przyst�pili�my do wyznaczenia czasu zdwojenia $T_i$. Po wielu testach zdecydowali�my si�, �e najlepszy przebieg oraz najni�sza warto�� b��du wyst�puje dla $T_i=3$. Cho� w przebiegu sterowania nie ma ona wi�kszej pzrewagi, to przebieg wyj�cia jest o wiele lepszy od konkurent�w. Na wykresie wyj�cie dos�ownie stapia si� w jeden przebieg z warto�ci� zadan�. Poni�ej w tabeli \ref{k1t1TiE1} przedstawiono warto�ci b��d�w dla r�nych warto�ci $T_i$. Przebiegi dla tych warto�ci pokazano na wykresie \ref{k1t1Ti}.

Ostatnim dobieranym parametrem by� czas wyprzedzenia $T_d$. Niemniej okaza�o si�, �e w��czenie cz�onu r�niczkowego powoduje bardziej pogorszenie przebiegu ni� jego polepszenie. Z tego powodu postanowili�my pozosta� przy warto�ci $T_d=0$. Warto�ci b��d�w dla wybranych warto�ci pzredstawione zosta�y w tabeli \ref{k1t1TdE1}, a przebiegi na wykresie \ref{k1t1Td}.

\begin{table}
	\centering
	\begin{tabular}{c|c}
		$\boldsymbol{T_i}$ & $\boldsymbol{E1}$ \\ \hline
		1 & 15,7016 \\
		3 & 14,0000 \\
		10 & 15,6587 \\
		100 & 34,8237 \\
	\end{tabular}
\caption{Warto�ci b��du dla r�nych warto�ci $T_i$}
\label{k1t1TiE1}
\end{table}

\begin{table}
	\centering
	\begin{tabular}{c|c}
		$\boldsymbol{T_i}$ & $\boldsymbol{E1}$ \\ \hline
		0 & 14 \\
		0.01 & 14,0113 \\
		0.1 & 15,5293 \\
	\end{tabular}
\caption{Warto�ci b��du dla r�nych warto�ci $T_d$}
\label{k1t1TdE1}
\end{table}
 
\begin{figure}[tb]
	\centering
	\begin{tikzpicture}
	\begin{groupplot}[group style={group size=1 by 2,vertical sep={1.5 cm}},
	width=0.9\textwidth,height=0.75\textwidth]
	\nextgroupplot
	[
	xmin=0,xmax=910,ymin=-50,ymax=50,
	xlabel={$k$},
	ylabel={$u3$},
	xtick={0,100,200,300,400,500,600,700,800,900},
	ytick={-50,-25,0,25,50},
	y tick label style={/pgf/number format/1000 sep=},
	]
	\addplot[blue,semithick] file {wykresy/pid_3_2_4/13.3642_0.0000_0.0000_Inf_Inf_Inf_0.0000_0.0000_0.0000/u3.txt};
	\nextgroupplot
	[
	xmin=0,xmax=910,ymin=-3,ymax=5,
	xlabel={$k$},
	ylabel={$y1$},
	xtick={0,100,200,300,400,500,600,700,800,900},
	ytick={-3,-2,-1,0,1,2,3,4,5},
	y tick label style={/pgf/number format/1000 sep=},
	]
	\addplot[blue,semithick] file {wykresy/pid_3_2_4/13.3642_0.0000_0.0000_Inf_Inf_Inf_0.0000_0.0000_0.0000/y1.txt};
	\addplot[red,semithick,densely dashed] file {wykresy/pid_3_2_4/13.3642_0.0000_0.0000_Inf_Inf_Inf_0.0000_0.0000_0.0000/yzad1.txt};
	\legend{$y1$,$y_{zad}$}
	\end{groupplot}
	\end{tikzpicture}
	\caption{Przebieg wyj�cia pierwszego i wej�cia trzeciego dla wzmocnienia oscylacyjnego $K_{osc}=13,3642$}
	\label{k1t1K}
\end{figure}

\begin{figure}[tb]
	\centering
	\begin{tikzpicture}
	\begin{groupplot}[group style={group size=1 by 2,vertical sep={1.5 cm}},
	width=0.9\textwidth,height=0.75\textwidth]
	\nextgroupplot
	[
	xmin=0,xmax=910,ymin=-50,ymax=50,
	xlabel={$k$},
	ylabel={$u3$},
	xtick={0,100,200,300,400,500,600,700,800,900},
	ytick={-50,-25,0,25,50},
	y tick label style={/pgf/number format/1000 sep=},
	]
	\addplot[blue,semithick] file {wykresy/pid_3_2_4/6.6821_0.0000_0.0000_1.0000_Inf_Inf_0.0000_0.0000_0.0000/u3.txt};
	\addplot[green,semithick] file {wykresy/pid_3_2_4/6.6821_0.0000_0.0000_3.0000_Inf_Inf_0.0000_0.0000_0.0000/u3.txt};
	\addplot[orange,semithick] file {wykresy/pid_3_2_4/6.6821_0.0000_0.0000_10.0000_Inf_Inf_0.0000_0.0000_0.0000/u3.txt};
	\addplot[brown,semithick] file {wykresy/pid_3_2_4/6.6821_0.0000_0.0000_100.0000_Inf_Inf_0.0000_0.0000_0.0000/u3.txt};
	\nextgroupplot
	[
	xmin=0,xmax=910,ymin=-3,ymax=5,
	xlabel={$k$},
	ylabel={$y1$},
	xtick={0,100,200,300,400,500,600,700,800,900},
	ytick={-3,-2,-1,0,1,2,3,4,5},
	y tick label style={/pgf/number format/1000 sep=},
	]
	\addplot[blue,semithick] file {wykresy/pid_3_2_4/6.6821_0.0000_0.0000_1.0000_Inf_Inf_0.0000_0.0000_0.0000/y1.txt};
	\addplot[green,semithick] file {wykresy/pid_3_2_4/6.6821_0.0000_0.0000_3.0000_Inf_Inf_0.0000_0.0000_0.0000/y1.txt};
	\addplot[orange,semithick] file {wykresy/pid_3_2_4/6.6821_0.0000_0.0000_10.0000_Inf_Inf_0.0000_0.0000_0.0000/y1.txt};
	\addplot[brown,semithick] file {wykresy/pid_3_2_4/6.6821_0.0000_0.0000_100.0000_Inf_Inf_0.0000_0.0000_0.0000/y1.txt};
	\addplot[red,semithick,densely dashed] file {wykresy/pid_3_2_4/13.3642_0.0000_0.0000_Inf_Inf_Inf_0.0000_0.0000_0.0000/yzad1.txt};
	\legend{$T_i=1$,$T_i=3$,$T_i=10$,$T_i=100$,$y_{zad}$}
	\end{groupplot}
	\end{tikzpicture}
	\caption{Przebieg wyj�cia pierwszego i wej�cia trzeciego dla r�nych warto�ci $T_i$}
	\label{k1t1Ti}
\end{figure}

\begin{figure}[tb]
	\centering
	\begin{tikzpicture}
	\begin{groupplot}[group style={group size=1 by 2,vertical sep={1.5 cm}},
	width=0.9\textwidth,height=0.75\textwidth]
	\nextgroupplot
	[
	xmin=0,xmax=910,ymin=-50,ymax=50,
	xlabel={$k$},
	ylabel={$u3$},
	xtick={0,100,200,300,400,500,600,700,800,900},
	ytick={-50,-25,0,25,50},
	y tick label style={/pgf/number format/1000 sep=},
	]
	\addplot[blue,semithick] file {wykresy/pid_3_2_4/6.6821_0.0000_0.0000_3.0000_Inf_Inf_0.1000_0.0000_0.0000/u3.txt};
	\addplot[green,semithick] file {wykresy/pid_3_2_4/6.6821_0.0000_0.0000_3.0000_Inf_Inf_0.0100_0.0000_0.0000/u3.txt};
	\addplot[orange,semithick] file {wykresy/pid_3_2_4/6.6821_0.0000_0.0000_3.0000_Inf_Inf_0.0000_0.0000_0.0000/u3.txt};
	\nextgroupplot
	[
	xmin=0,xmax=910,ymin=-3,ymax=5,
	xlabel={$k$},
	ylabel={$y1$},
	xtick={0,100,200,300,400,500,600,700,800,900},
	ytick={-3,-2,-1,0,1,2,3,4,5},
	y tick label style={/pgf/number format/1000 sep=},
	]
	\addplot[blue,semithick] file {wykresy/pid_3_2_4/6.6821_0.0000_0.0000_3.0000_Inf_Inf_0.1000_0.0000_0.0000/y1.txt};
	\addplot[green,semithick] file {wykresy/pid_3_2_4/6.6821_0.0000_0.0000_3.0000_Inf_Inf_0.0100_0.0000_0.0000/y1.txt};
	\addplot[orange,semithick] file {wykresy/pid_3_2_4/6.6821_0.0000_0.0000_3.0000_Inf_Inf_0.0000_0.0000_0.0000/y1.txt};
	\addplot[red,semithick,densely dashed] file {wykresy/pid_3_2_4/13.3642_0.0000_0.0000_Inf_Inf_Inf_0.0000_0.0000_0.0000/yzad1.txt};
	\legend{$T_d=\num{0,1}$,$T_d=\num{0,01}$,$T_d=0$,$y_{zad}$}
	\end{groupplot}
	\end{tikzpicture}
	\caption{Przebieg wyj�cia pierwszego i wej�cia trzeciego dla r�nych warto�ci $T_d$}
	\label{k1t1Td}
\end{figure}
\FloatBarrier
\subsubsection{Tor drugi}
Oczywi�cie zaczynamy od dobrania wzmocnienia K. Dla wyj�cia drugiego sterowanego drugim wej�ciem warto�� wzmocnienie dla kt�rego wpada ono w oscylacje to $K_{osc}$=18,197. Oznacza to, �e dla regulatora przyjmiemy wzmocnienie r�wne K=9,0985. Wykres zawieraj�cy przebiegi dla wzmocnienia oscylacyjnego umie�cili�my pod numerem \ref{k1t2K}.

Nast�pn� dobieran� warto�ci� by�o $T_i$. Po kilku pr�bach doszli�my do wniosku, �e najlepsz� warto�ci� jest $T_i$=5. Dla tej warto�ci zar�wno przebieg jak i warto�� b��du s� najmniejsze. B��dy dla r�nych warto�ci znajduj� si� w tabeli \ref{k1t2TiE2}. Przebiegi zamie�cili�my na wykresie \ref{k1t2Ti}.

Ostatnim dobieranym parametrem by� czas wyprzedzenia $T_d$. Niemniej okaza�o si�, �e w��czenie cz�onu r�niczkowego powoduje bardziej pogorszenie przebiegu ni� jego polepszenie. Z tego powodu postanowili�my pozosta� przy warto�ci $T_d=0$. Warto�ci b��d�w dla wybranych warto�ci przedstawione zosta�y w tabeli \ref{k1t2TdE2}, a przebiegi na wykresie \ref{k1t2Td}.

\begin{table}
	\centering
	\begin{tabular}{c|c}
		$\boldsymbol{T_i}$ & $\boldsymbol{E1}$ \\ \hline
		1 & 16,6367 \\
		5 & 14,0000 \\
		10 & 14,3245 \\
		100 & 21,2770 \\
	\end{tabular}
	\caption{Warto�ci b��du dla r�nych warto�ci $T_i$}
	\label{k1t2TiE2}
\end{table}

\begin{table}
	\centering
	\begin{tabular}{c|c}
		$\boldsymbol{T_i}$ & $\boldsymbol{E1}$ \\ \hline
		0 & 14 \\
		0.01 & 14,0114 \\
		0.1 & 15,5409 \\
	\end{tabular}
	\caption{Warto�ci b��du dla r�nych warto�ci $T_d$}
	\label{k1t2TdE2}
\end{table}

\begin{figure}[tb]
	\centering
	\begin{tikzpicture}
	\begin{groupplot}[group style={group size=1 by 2,vertical sep={1.5 cm}},
	width=0.9\textwidth,height=0.75\textwidth]
	\nextgroupplot
	[
	xmin=0,xmax=910,ymin=-50,ymax=50,
	xlabel={$k$},
	ylabel={$u2$},
	xtick={0,100,200,300,400,500,600,700,800,900},
	ytick={-50,-25,0,25,50},
	y tick label style={/pgf/number format/1000 sep=},
	]
	\addplot[blue,semithick] file {wykresy/pid_3_2_4/0.0000_18.1970_0.0000_Inf_Inf_Inf_0.0000_0.0000_0.0000/u2.txt};
	\nextgroupplot
	[
	xmin=0,xmax=910,ymin=-3,ymax=5,
	xlabel={$k$},
	ylabel={$y2$},
	xtick={0,100,200,300,400,500,600,700,800,900},
	ytick={-3,-2,-1,0,1,2,3,4,5},
	y tick label style={/pgf/number format/1000 sep=},
	]
	\addplot[blue,semithick] file {wykresy/pid_3_2_4/0.0000_18.1970_0.0000_Inf_Inf_Inf_0.0000_0.0000_0.0000/y2.txt};
	\addplot[red,semithick,densely dashed] file {wykresy/pid_3_2_4/13.3642_0.0000_0.0000_Inf_Inf_Inf_0.0000_0.0000_0.0000/yzad2.txt};
	\legend{$y1$,$y_{zad}$}
	\end{groupplot}
	\end{tikzpicture}
	\caption{Przebieg wyj�cia drugiego i wej�cia drugiego dla wzmocnienia oscylacyjnego $K_{osc}=18,197$}
	\label{k1t2K}
\end{figure}

\begin{figure}[tb]
	\centering
	\begin{tikzpicture}
	\begin{groupplot}[group style={group size=1 by 2,vertical sep={1.5 cm}},
	width=0.9\textwidth,height=0.75\textwidth]
	\nextgroupplot
	[
	xmin=0,xmax=910,ymin=-50,ymax=50,
	xlabel={$k$},
	ylabel={$u2$},
	xtick={0,100,200,300,400,500,600,700,800,900},
	ytick={-50,-25,0,25,50},
	y tick label style={/pgf/number format/1000 sep=},
	]
	\addplot[blue,semithick] file {wykresy/pid_3_2_4/0.0000_9.0985_0.0000_Inf_1.0000_Inf_0.0000_0.0000_0.0000/u2.txt};
	\addplot[green,semithick] file {wykresy/pid_3_2_4/0.0000_9.0985_0.0000_Inf_5.0000_Inf_0.0000_0.0000_0.0000/u2.txt};
	\addplot[orange,semithick] file {wykresy/pid_3_2_4/0.0000_9.0985_0.0000_Inf_10.0000_Inf_0.0000_0.0000_0.0000/u2.txt};
	\addplot[brown,semithick] file {wykresy/pid_3_2_4/0.0000_9.0985_0.0000_Inf_100.0000_Inf_0.0000_0.0000_0.0000/u2.txt};
	\nextgroupplot
	[
	xmin=0,xmax=910,ymin=-3,ymax=5,
	xlabel={$k$},
	ylabel={$y2$},
	xtick={0,100,200,300,400,500,600,700,800,900},
	ytick={-3,-2,-1,0,1,2,3,4,5},
	y tick label style={/pgf/number format/1000 sep=},
	]
	\addplot[blue,semithick] file {wykresy/pid_3_2_4/0.0000_9.0985_0.0000_Inf_1.0000_Inf_0.0000_0.0000_0.0000/y2.txt};
	\addplot[green,semithick] file {wykresy/pid_3_2_4/0.0000_9.0985_0.0000_Inf_5.0000_Inf_0.0000_0.0000_0.0000/y2.txt};
	\addplot[orange,semithick] file {wykresy/pid_3_2_4/0.0000_9.0985_0.0000_Inf_10.0000_Inf_0.0000_0.0000_0.0000/y2.txt};
	\addplot[brown,semithick] file {wykresy/pid_3_2_4/0.0000_9.0985_0.0000_Inf_100.0000_Inf_0.0000_0.0000_0.0000/y2.txt};
	\addplot[red,semithick,densely dashed] file {wykresy/pid_3_2_4/0.0000_9.0985_0.0000_Inf_5.0000_Inf_0.0000_0.0000_0.0000/yzad2.txt};
	\legend{$T_i=1$,$T_i=5$,$T_i=10$,$T_i=100$,$y_{zad}$}
	\end{groupplot}
	\end{tikzpicture}
	\caption{Przebieg wyj�cia drugiego i wej�cia drugiego dla r�nych warto�ci $T_i$}
	\label{k1t2Ti}
\end{figure}

\begin{figure}[tb]
	\centering
	\begin{tikzpicture}
	\begin{groupplot}[group style={group size=1 by 2,vertical sep={1.5 cm}},
	width=0.9\textwidth,height=0.75\textwidth]
	\nextgroupplot
	[
	xmin=0,xmax=910,ymin=-50,ymax=50,
	xlabel={$k$},
	ylabel={$u2$},
	xtick={0,100,200,300,400,500,600,700,800,900},
	ytick={-50,-25,0,25,50},
	y tick label style={/pgf/number format/1000 sep=},
	]
	\addplot[blue,semithick] file {wykresy/pid_3_2_4/0.0000_9.0985_0.0000_Inf_5.0000_Inf_0.0000_0.1000_0.0000/u2.txt};
	\addplot[green,semithick] file {wykresy/pid_3_2_4/0.0000_9.0985_0.0000_Inf_5.0000_Inf_0.0000_0.0100_0.0000/u2.txt};
	\addplot[orange,semithick] file {wykresy/pid_3_2_4/0.0000_9.0985_0.0000_Inf_5.0000_Inf_0.0000_0.0000_0.0000/u2.txt};
	\nextgroupplot
	[
	xmin=0,xmax=910,ymin=-3,ymax=5,
	xlabel={$k$},
	ylabel={$y2$},
	xtick={0,100,200,300,400,500,600,700,800,900},
	ytick={-3,-2,-1,0,1,2,3,4,5},
	y tick label style={/pgf/number format/1000 sep=},
	]
	\addplot[blue,semithick] file {wykresy/pid_3_2_4/0.0000_9.0985_0.0000_Inf_5.0000_Inf_0.0000_0.1000_0.0000/y2.txt};
	\addplot[green,semithick] file {wykresy/pid_3_2_4/0.0000_9.0985_0.0000_Inf_5.0000_Inf_0.0000_0.0100_0.0000/y2.txt};
	\addplot[orange,semithick] file {wykresy/pid_3_2_4/0.0000_9.0985_0.0000_Inf_5.0000_Inf_0.0000_0.0000_0.0000/y2.txt};
	\addplot[red,semithick,densely dashed] file {wykresy/pid_3_2_4/0.0000_9.0985_0.0000_Inf_5.0000_Inf_0.0000_0.0000_0.0000/yzad2.txt};
	\legend{$T_d=\num{0,1}$,$T_d=\num{0,01}$,$T_d=0$,$y_{zad}$}
	\end{groupplot}
	\end{tikzpicture}
	\caption{Przebieg wyj�cia drugiego i wej�cia drugiego dla r�nych warto�ci $T_d$}
	\label{k1t2Td}
\end{figure}
\FloatBarrier

\subsubsection{Tor trzeci}
Zaczynamy od wyznaczenia wzmocnienia K. Dla trzeciego toru $K_{osc}=10,459$, co oznacza, �e jako warto�� K przyjmujemy $K=5,2295$. Oscylacje przedstawia poni�szy wykres \ref{k1t3K}.

Nast�pnie przyst�pili�my do wyznaczenia czasu zdwojenia $T_i$. Po wielu testach zdecydowali�my si�, �e najlepszy przebieg oraz najni�sza warto�� b��du wyst�puje dla $T_i=3$. Cho� w przebiegu sterowania nie ma ona wi�kszej pzrewagi, to przebieg wyj�cia jest o wiele lepszy od konkurent�w. Na wykresie wyj�cie dos�ownie stapia si� w jeden przebieg z warto�ci� zadan�. Poni�ej w tabeli \ref{k1t3TiE3} przedstawiono warto�ci b��d�w dla r�nych warto�ci $T_i$. Przebiegi dla tych warto�ci pokazano na wykresie \ref{k1t3Ti}.

Ostatnim dobieranym parametrem by� czas wyprzedzenia $T_d$. Niemniej okaza�o si�, �e w��czenie cz�onu r�niczkowego powoduje bardziej pogorszenie przebiegu ni� jego polepszenie. Z tego powodu postanowili�my pozosta� przy warto�ci $T_d=0$. Warto�ci b��d�w dla wybranych warto�ci pzredstawione zosta�y w tabeli \ref{k1t3TdE3}, a przebiegi na wykresie \ref{k1t3Td}.

\begin{table}
	\centering
	\begin{tabular}{c|c}
		$\boldsymbol{T_i}$ & $\boldsymbol{E3}$ \\ \hline
		1 & 15,7016 \\
		3 & 14,0000 \\
		10 & 15,6585 \\
		100 & 29,845 \\
	\end{tabular}
	\caption{Warto�ci b��du dla r�nych warto�ci $T_i$}
	\label{k1t3TiE3}
\end{table}

\begin{table}
	\centering
	\begin{tabular}{c|c}
		$\boldsymbol{T_i}$ & $\boldsymbol{E3}$ \\ \hline
		0 & 14 \\
		0.01 & 14,0113 \\
		0.1 & 15,5294 \\
	\end{tabular}
	\caption{Warto�ci b��du dla r�nych warto�ci $T_d$}
	\label{k1t3TdE3}
\end{table}

\begin{figure}[tb]
	\centering
	\begin{tikzpicture}
	\begin{groupplot}[group style={group size=1 by 2,vertical sep={1.5 cm}},
	width=0.9\textwidth,height=0.75\textwidth]
	\nextgroupplot
	[
	xmin=0,xmax=910,ymin=-50,ymax=50,
	xlabel={$k$},
	ylabel={$u2$},
	xtick={0,100,200,300,400,500,600,700,800,900},
	ytick={-50,-25,0,25,50},
	y tick label style={/pgf/number format/1000 sep=},
	]
	\addplot[blue,semithick] file {wykresy/pid_3_2_4/0.0000_0.0000_10.4590_Inf_Inf_Inf_0.0000_0.0000_0.0000/u4.txt};
	\nextgroupplot
	[
	xmin=0,xmax=910,ymin=-3,ymax=5,
	xlabel={$k$},
	ylabel={$y2$},
	xtick={0,100,200,300,400,500,600,700,800,900},
	ytick={-3,-2,-1,0,1,2,3,4,5},
	y tick label style={/pgf/number format/1000 sep=},
	]
	\addplot[blue,semithick] file {wykresy/pid_3_2_4/0.0000_0.0000_10.4590_Inf_Inf_Inf_0.0000_0.0000_0.0000/y3.txt};
	\addplot[red,semithick,densely dashed] file {wykresy/pid_3_2_4/0.0000_0.0000_10.4590_Inf_Inf_Inf_0.0000_0.0000_0.0000/yzad3.txt};
	\legend{$y1$,$y_{zad}$}
	\end{groupplot}
	\end{tikzpicture}
	\caption{Przebieg wyj�cia trzeciego i wej�cia czwartego dla wzmocnienia oscylacyjnego $K_{osc}=10,459$}
	\label{k1t3K}
\end{figure}

\begin{figure}[tb]
	\centering
	\begin{tikzpicture}
	\begin{groupplot}[group style={group size=1 by 2,vertical sep={1.5 cm}},
	width=0.9\textwidth,height=0.75\textwidth]
	\nextgroupplot
	[
	xmin=0,xmax=910,ymin=-50,ymax=50,
	xlabel={$k$},
	ylabel={$u2$},
	xtick={0,100,200,300,400,500,600,700,800,900},
	ytick={-50,-25,0,25,50},
	y tick label style={/pgf/number format/1000 sep=},
	]
	\addplot[blue,semithick] file {wykresy/pid_3_2_4/0.0000_0.0000_5.2295_Inf_Inf_1.0000_0.0000_0.0000_0.0000/u4.txt};
	\addplot[green,semithick] file {wykresy/pid_3_2_4/0.0000_0.0000_5.2295_Inf_Inf_3.0000_0.0000_0.0000_0.0000/u4.txt};
	\addplot[orange,semithick] file {wykresy/pid_3_2_4/0.0000_0.0000_5.2295_Inf_Inf_10.0000_0.0000_0.0000_0.0000/u4.txt};
	\addplot[brown,semithick] file {wykresy/pid_3_2_4/0.0000_0.0000_5.2295_Inf_Inf_100.0000_0.0000_0.0000_0.0000/u4.txt};
	\nextgroupplot
	[
	xmin=0,xmax=910,ymin=-3,ymax=5,
	xlabel={$k$},
	ylabel={$y2$},
	xtick={0,100,200,300,400,500,600,700,800,900},
	ytick={-3,-2,-1,0,1,2,3,4,5},
	y tick label style={/pgf/number format/1000 sep=},
	]
	\addplot[blue,semithick] file {wykresy/pid_3_2_4/0.0000_0.0000_5.2295_Inf_Inf_1.0000_0.0000_0.0000_0.0000/y3.txt};
	\addplot[green,semithick] file {wykresy/pid_3_2_4/0.0000_0.0000_5.2295_Inf_Inf_3.0000_0.0000_0.0000_0.0000/y3.txt};
	\addplot[orange,semithick] file {wykresy/pid_3_2_4/0.0000_0.0000_5.2295_Inf_Inf_10.0000_0.0000_0.0000_0.0000/y3.txt};
	\addplot[brown,semithick] file {wykresy/pid_3_2_4/0.0000_0.0000_5.2295_Inf_Inf_100.0000_0.0000_0.0000_0.0000/y3.txt};
	\addplot[red,semithick,densely dashed] file {wykresy/pid_3_2_4/0.0000_0.0000_5.2295_Inf_Inf_1.0000_0.0000_0.0000_0.0000/yzad3.txt};
	\legend{$T_i=1$,$T_i=3$,$T_i=10$,$T_i=100$,$y_{zad}$}
	\end{groupplot}
	\end{tikzpicture}
	\caption{Przebieg wyj�cia trzeciego i wej�cia czwartego dla r�nych warto�ci $T_i$}
	\label{k1t3Ti}
\end{figure}

\begin{figure}[tb]
	\centering
	\begin{tikzpicture}
	\begin{groupplot}[group style={group size=1 by 2,vertical sep={1.5 cm}},
	width=0.9\textwidth,height=0.75\textwidth]
	\nextgroupplot
	[
	xmin=0,xmax=910,ymin=-50,ymax=50,
	xlabel={$k$},
	ylabel={$u2$},
	xtick={0,100,200,300,400,500,600,700,800,900},
	ytick={-50,-25,0,25,50},
	y tick label style={/pgf/number format/1000 sep=},
	]
	\addplot[blue,semithick] file {wykresy/pid_3_2_4/0.0000_0.0000_5.2295_Inf_Inf_3.0000_0.0000_0.0000_0.1000/u4.txt};
	\addplot[green,semithick] file {wykresy/pid_3_2_4/0.0000_0.0000_5.2295_Inf_Inf_3.0000_0.0000_0.0000_0.0100/u4.txt};
	\addplot[orange,semithick] file {wykresy/pid_3_2_4/0.0000_0.0000_5.2295_Inf_Inf_3.0000_0.0000_0.0000_0.0000/u4.txt};
	\nextgroupplot
	[
	xmin=0,xmax=910,ymin=-3,ymax=5,
	xlabel={$k$},
	ylabel={$y2$},
	xtick={0,100,200,300,400,500,600,700,800,900},
	ytick={-3,-2,-1,0,1,2,3,4,5},
	y tick label style={/pgf/number format/1000 sep=},
	]
	\addplot[blue,semithick] file {wykresy/pid_3_2_4/0.0000_0.0000_5.2295_Inf_Inf_3.0000_0.0000_0.0000_0.1000/y3.txt};
	\addplot[green,semithick] file {wykresy/pid_3_2_4/0.0000_0.0000_5.2295_Inf_Inf_3.0000_0.0000_0.0000_0.0100/y3.txt};
	\addplot[orange,semithick] file {wykresy/pid_3_2_4/0.0000_0.0000_5.2295_Inf_Inf_3.0000_0.0000_0.0000_0.0000/y3.txt};
	\addplot[red,semithick,densely dashed] file {wykresy/pid_3_2_4/0.0000_0.0000_5.2295_Inf_Inf_3.0000_0.0000_0.0000_0.0000/yzad3.txt};
	\legend{$T_d=\num{0,1}$,$T_d=\num{0,01}$,$T_d=0$,$y_{zad}$}
	\end{groupplot}
	\end{tikzpicture}
	\caption{Przebieg wyj�cia trzeciego i wej�cia czwartego dla r�nych warto�ci $T_d$}
	\label{k1t3Td}
\end{figure}
\FloatBarrier
\subsubsection{Ca�o��}
Poni�ej na wykresach \ref{k1u1} oraz \ref{k1y1} znajduje si� przebieg dzia�ania ca�ego procesu (wszystkie tory aktywne, trzy sterowania w odpowiedniej kolejno�ci i trzy wyj�cia) dla nastaw wybranych w poprzednich podpunktach. Warto�ci b��d�w znajduj� si� w tabeli \ref{k1E1}. Jak mo�na zauwa�y� przebiegi nie s� idealne, zw�aszcza dla toru pierwszego z kt�rym wyst�puj� znaczne odchylenia. Fakt, �e nastawy nale�y poprawi� nie by� dla nas zaskoczeniem, w ko�cu ka�de sterowanie ma wp�yw na ka�de wyj�cie. Po wielu pr�bach doszli�my do nast�puj�cych wniosk�w:
\begin{itemize}
	\item w��czenie kt�regokolwiek cz�onu D pogarsza regulacj�
	\item zmienianie czasu $T_I$ cz�on�w ma bardzo ma�y wp�yw na popraw� regulacji i cz�sto j� pogarsza
	\item najlepszy wp�yw na regulacj� ma zmniejszanie wzmocnie� cz�on�w 
\end{itemize} 

Ostatecznie zdecydowali�my si� pozostawi� warto�ci $T_i$ i $T_d$ cz�on�w niezmienione oraz zmniejszy� wzmocnienia toru drugiego i trzeciego (sterowania u2 i u4) trzykrotnie. Pow�d dla kt�rego nie zmienili�my wzmocnienie pierwszego z tor�w wynika z tabeli wzmocnie� znajduj�cej si� na pocz�tku sekcji PID. Wyra�nie wida� w niej, �e u3 ma ma�y wp�yw na wyj�cia y2 oraz y3. Z drugiej strony u2 i u4 maj� bardzo du�y wp�yw na y1. Wzmocnienia dla wszystkich tor�w po poprawkach przyjmuj� warto�ci $K_1=6,6821$, $K_2=4,4592$, $K_3=2,6147$. Poni�ej w tabeli \ref{k1E2} znajduj� si� wyliczone nowe warto�ci b��d�w. Du�� popraw� wida� zw�aszcza dla toru pierwszego. B��d dla y2 nieco si� powi�kszy�, ale mimo to i tak jest najmniejszy ze wszystkich. Przebiegi sterowa� i wyj�� dla poprawionych nastaw znajduj� si� na wykresach \ref{k1u2} oraz \ref{k1y2}. Tu tak�e wida� ogromn� popraw� przebiegu regulacji, zw��szcza dla wyj�cia pierwszego, ale nie tylko.

\begin{table}
	\centering
	\begin{tabular}{c|c|c|c}
		$\boldsymbol{E}$ & $\boldsymbol{E1}$ & $\boldsymbol{E2}$ & $\boldsymbol{E3}$ \\ \hline
		284,3723 & 242,0791 & 15,0974 & 27,1958 \\
	\end{tabular}
	\caption{Warto�ci b��d�w dla nastaw oryginalnych}
	\label{k1E1}
\end{table}

\begin{table}
	\centering
	\begin{tabular}{c|c|c|c}
		$\boldsymbol{E}$ & $\boldsymbol{E1}$ & $\boldsymbol{E2}$ & $\boldsymbol{E3}$ \\ \hline
		93,5619 & 47,5939 & 19,5724 & 26,3956 \\
	\end{tabular}
	\caption{Warto�ci b��d�w dla nastaw poprawionych}
	\label{k1E2}
\end{table}

\begin{figure}[tb]
	\centering
	\begin{tikzpicture}
	\begin{groupplot}[group style={group size=1 by 3,vertical sep={1.5 cm}},
	width=0.9\textwidth,height=0.4\textwidth]
	\nextgroupplot
	[
	xmin=0,xmax=910,ymin=-50,ymax=50,
	xlabel={$k$},
	ylabel={$u3$},
	xtick={0,100,200,300,400,500,600,700,800,900},
	ytick={-50,-25,0,25,50},
	y tick label style={/pgf/number format/1000 sep=},
	]
	\addplot[blue,semithick] file {wykresy/pid_3_2_4/6.6821_9.0985_5.2295_3.0000_5.0000_3.0000_0.0000_0.0000_0.0000/u3.txt};
	\nextgroupplot
	[
	xmin=0,xmax=910,ymin=-50,ymax=50,
	xlabel={$k$},
	ylabel={$u2$},
	xtick={0,100,200,300,400,500,600,700,800,900},
	ytick={-50,-25,0,25,50},
	y tick label style={/pgf/number format/1000 sep=},
	]
	\addplot[blue,semithick] file {wykresy/pid_3_2_4/6.6821_9.0985_5.2295_3.0000_5.0000_3.0000_0.0000_0.0000_0.0000/u2.txt};
	\nextgroupplot
	[
	xmin=0,xmax=910,ymin=-50,ymax=50,
	xlabel={$k$},
	ylabel={$u4$},
	xtick={0,100,200,300,400,500,600,700,800,900},
	ytick={-50,-25,0,25,50},
	y tick label style={/pgf/number format/1000 sep=},
	]
	\addplot[blue,semithick] file {wykresy/pid_3_2_4/6.6821_9.0985_5.2295_3.0000_5.0000_3.0000_0.0000_0.0000_0.0000/u4.txt};

	\end{groupplot}
	\end{tikzpicture}
	\caption{Przebiegi sterowa� dla oryginalnych nastaw}
	\label{k1u1}
\end{figure}

\begin{figure}[tb]
	\centering
	\begin{tikzpicture}
	\begin{groupplot}[group style={group size=1 by 3,vertical sep={1.5 cm}},
	width=0.9\textwidth,height=0.4\textwidth]
	\nextgroupplot
	[
	xmin=0,xmax=910,ymin=-10,ymax=10,
	xlabel={$k$},
	ylabel={$y1$},
	xtick={0,100,200,300,400,500,600,700,800,900},
	ytick={-10,-5,0,5,10},
	y tick label style={/pgf/number format/1000 sep=},
	legend pos=south east,
	]
	\addplot[blue,semithick] file {wykresy/pid_3_2_4/6.6821_9.0985_5.2295_3.0000_5.0000_3.0000_0.0000_0.0000_0.0000/y1.txt};
	\addplot[red,semithick,densely dashed] file {wykresy/pid_3_2_4/6.6821_9.0985_5.2295_3.0000_5.0000_3.0000_0.0000_0.0000_0.0000/yzad1.txt};
	\legend{$y1$,$y1_{zad}$}
	\nextgroupplot
	[
	xmin=0,xmax=910,ymin=-10,ymax=10,
	xlabel={$k$},
	ylabel={$y2$},
	xtick={0,100,200,300,400,500,600,700,800,900},
	ytick={-10,-5,0,5,10},
	y tick label style={/pgf/number format/1000 sep=},
	legend pos=south east,
	]
	\addplot[blue,semithick] file {wykresy/pid_3_2_4/6.6821_9.0985_5.2295_3.0000_5.0000_3.0000_0.0000_0.0000_0.0000/y2.txt};
	\addplot[red,semithick,densely dashed] file {wykresy/pid_3_2_4/6.6821_9.0985_5.2295_3.0000_5.0000_3.0000_0.0000_0.0000_0.0000/yzad2.txt};
	\legend{$y1$,$y1_{zad}$}
	\nextgroupplot
	[
	xmin=0,xmax=910,ymin=-10,ymax=10,
	xlabel={$k$},
	ylabel={$y3$},
	xtick={0,100,200,300,400,500,600,700,800,900},
	ytick={-10,-5,0,5,10},
	y tick label style={/pgf/number format/1000 sep=},
	legend pos=south east,
	]
	\addplot[blue,semithick] file {wykresy/pid_3_2_4/6.6821_9.0985_5.2295_3.0000_5.0000_3.0000_0.0000_0.0000_0.0000/y3.txt};
	\addplot[red,semithick,densely dashed] file {wykresy/pid_3_2_4/6.6821_9.0985_5.2295_3.0000_5.0000_3.0000_0.0000_0.0000_0.0000/yzad3.txt};
	\legend{$y1$,$y1_{zad}$}
	\end{groupplot}
	\end{tikzpicture}
	\caption{Przebiegi wyj�� dla oryginalnych nastaw}
	\label{k1y1}
\end{figure}

\begin{figure}[tb]
	\centering
	\begin{tikzpicture}
	\begin{groupplot}[group style={group size=1 by 3,vertical sep={1.5 cm}},
	width=0.9\textwidth,height=0.4\textwidth]
	\nextgroupplot
	[
	xmin=0,xmax=910,ymin=-50,ymax=50,
	xlabel={$k$},
	ylabel={$u3$},
	xtick={0,100,200,300,400,500,600,700,800,900},
	ytick={-50,-25,0,25,50},
	y tick label style={/pgf/number format/1000 sep=},
	]
	\addplot[blue,semithick] file {wykresy/pid_3_2_4/6.6821_4.5492_2.6147_3.0000_5.0000_3.0000_0.0000_0.0000_0.0000/u3.txt};
	\nextgroupplot
	[
	xmin=0,xmax=910,ymin=-50,ymax=50,
	xlabel={$k$},
	ylabel={$u2$},
	xtick={0,100,200,300,400,500,600,700,800,900},
	ytick={-50,-25,0,25,50},
	y tick label style={/pgf/number format/1000 sep=},
	]
	\addplot[blue,semithick] file {wykresy/pid_3_2_4/6.6821_4.5492_2.6147_3.0000_5.0000_3.0000_0.0000_0.0000_0.0000/u2.txt};
	\nextgroupplot
	[
	xmin=0,xmax=910,ymin=-50,ymax=50,
	xlabel={$k$},
	ylabel={$u4$},
	xtick={0,100,200,300,400,500,600,700,800,900},
	ytick={-50,-25,0,25,50},
	y tick label style={/pgf/number format/1000 sep=},
	]
	\addplot[blue,semithick] file {wykresy/pid_3_2_4/6.6821_4.5492_2.6147_3.0000_5.0000_3.0000_0.0000_0.0000_0.0000/u4.txt};
	
	\end{groupplot}
	\end{tikzpicture}
	\caption{Przebiegi sterowa� dla poprawionych nastaw}
	\label{k1u2}
\end{figure}

\begin{figure}[tb]
	\centering
	\begin{tikzpicture}
	\begin{groupplot}[group style={group size=1 by 3,vertical sep={1.5 cm}},
	width=0.9\textwidth,height=0.4\textwidth]
	\nextgroupplot
	[
	xmin=0,xmax=910,ymin=-10,ymax=10,
	xlabel={$k$},
	ylabel={$y1$},
	xtick={0,100,200,300,400,500,600,700,800,900},
	ytick={-10,-5,0,5,10},
	y tick label style={/pgf/number format/1000 sep=},
	legend pos=south east,
	]
	\addplot[blue,semithick] file {wykresy/pid_3_2_4/6.6821_4.5492_2.6147_3.0000_5.0000_3.0000_0.0000_0.0000_0.0000/y1.txt};
	\addplot[red,semithick,densely dashed] file {wykresy/pid_3_2_4/6.6821_4.5492_2.6147_3.0000_5.0000_3.0000_0.0000_0.0000_0.0000/yzad1.txt};
	\legend{$y1$,$y1_{zad}$}
	\nextgroupplot
	[
	xmin=0,xmax=910,ymin=-10,ymax=10,
	xlabel={$k$},
	ylabel={$y2$},
	xtick={0,100,200,300,400,500,600,700,800,900},
	ytick={-10,-5,0,5,10},
	y tick label style={/pgf/number format/1000 sep=},
	legend pos=south east,
	]
	\addplot[blue,semithick] file {wykresy/pid_3_2_4/6.6821_4.5492_2.6147_3.0000_5.0000_3.0000_0.0000_0.0000_0.0000/y2.txt};
	\addplot[red,semithick,densely dashed] file {wykresy/pid_3_2_4/6.6821_4.5492_2.6147_3.0000_5.0000_3.0000_0.0000_0.0000_0.0000/yzad2.txt};
	\legend{$y1$,$y1_{zad}$}
	\nextgroupplot
	[
	xmin=0,xmax=910,ymin=-10,ymax=10,
	xlabel={$k$},
	ylabel={$y3$},
	xtick={0,100,200,300,400,500,600,700,800,900},
	ytick={-10,-5,0,5,10},
	y tick label style={/pgf/number format/1000 sep=},
	legend pos=south east,
	]
	\addplot[blue,semithick] file {wykresy/pid_3_2_4/6.6821_4.5492_2.6147_3.0000_5.0000_3.0000_0.0000_0.0000_0.0000/y3.txt};
	\addplot[red,semithick,densely dashed] file {wykresy/pid_3_2_4/6.6821_4.5492_2.6147_3.0000_5.0000_3.0000_0.0000_0.0000_0.0000/yzad3.txt};
	\legend{$y1$,$y1_{zad}$}
	\end{groupplot}
	\end{tikzpicture}
	\caption{Przebiegi wyj�� dla poprawionych nastaw}
	\label{k1y2}
\end{figure}
\FloatBarrier
\subsection{PID - konfiguracja druga}
Pierwsza konfiguracja naszego PID'a zak�ada, �e wyj�cie pierwsze sterujemy wej�ciem trzecim, wyj�cie drugie wej�ciem pierwszym, a wyj�cie trzecie wej�ciem czwartym.
\begin{table}
	\centering
	\begin{tabular}{c|c}
		$\boldsymbol{y}$ & $\boldsymbol{u}$\\ \hline
		$y_1$ & $u_3$ \\
		$y_2$ & $u_1$ \\
		$y_3$ & $u_4$ 
		\label{k2}
	\end{tabular}
	\caption{Druga konfiguracja}
\end{table}
%\chapter{Zadanie 5: Automatyczne dobieranie nastaw}
W tym zadaniu naszym zadaniem by�o automatyczne dobranie nastaw regulator�w w wyniku optymalizacji wska�nika jako�ci. U�yli�my w tym celu wbudowanej w pakiet Matlab funkcji fmincon, podaj�c jako pocz�tkowe parametry wyliczone przez nas konfiguracje.
\section{PID}
\subsection{Konfiguracja pierwsza}
Jak widzimy w tym przypadku wyliczone przez funkcj� parametry s� podobne do naszych, aczkolwiek daj� nieco lepsze rozwi�zanie pod k�tem wielko�ci wska�nika regulacji. Wizualnie przebiegi prawie si� nie r�ni�.
\begin{table}
	\centering
	\begin{tabular}{|c|c|c|c|}
		\hline
		& $\boldsymbol{y1}$ & $\boldsymbol{y2}$ & $\boldsymbol{y3}$ \\ \hline
		$\boldsymbol{K}$ & 6,6821 & 4,4592 & 2,6147 \\ \hline
		$\boldsymbol{T_i}$ & 3 & 5 & 3 \\ \hline
		$\boldsymbol{T_d}$ & 0 & 0 & 0 \\ \hline
	\end{tabular}
	\caption{Nasze nastawy}
	\label{pid1n}

	\centering
	\begin{tabular}{c|c|c|c}
		$\boldsymbol{E}$ & $\boldsymbol{E1}$ & $\boldsymbol{E2}$ & $\boldsymbol{E3}$ \\ \hline
		93,5137 & 47,3773 & 19,8595 & 26,2770 \\
	\end{tabular}
	\caption{Warto�ci b��d�w dla naszych nastaw}
	\label{pid1e1}
\end{table}

\begin{table}
	\centering
	\begin{tabular}{|c|c|c|c|}
		\hline
		& $\boldsymbol{y1}$ & $\boldsymbol{y2}$ & $\boldsymbol{y3}$ \\ \hline
		$\boldsymbol{K}$ & 7,8600 & 4,2845 & 1,6460 \\ \hline
		$\boldsymbol{T_i}$ & 2,1371 & 4,3444 & 1,8913 \\ \hline
		$\boldsymbol{T_d}$ & 9,0673e-9 & 0,0132 & 1,4253e-7 \\ \hline
	\end{tabular}
	\caption{Wyliczone nastawy}
	\label{pid1n2}
\end{table}

\begin{table}
	\centering
	\begin{tabular}{c|c|c|c}
		$\boldsymbol{E}$ & $\boldsymbol{E1}$ & $\boldsymbol{E2}$ & $\boldsymbol{E3}$ \\ \hline
		85,3434 & 30,8617 & 19,8453 & 34,6364 \\
	\end{tabular}
	\caption{Warto�ci b��d�w dla wyliczonych nastaw}
	\label{pid1e2}
\end{table}

\begin{figure}[tb]
	\centering
	\begin{tikzpicture}
	\begin{groupplot}[group style={group size=1 by 3,vertical sep={1.5 cm}},
	width=0.9\textwidth,height=0.4\textwidth]
	\nextgroupplot
	[
	xmin=0,xmax=910,ymin=-50,ymax=50,
	xlabel={$k$},
	ylabel={$u3$},
	xtick={0,100,200,300,400,500,600,700,800,900},
	ytick={-50,-25,0,25,50},
	y tick label style={/pgf/number format/1000 sep=},
	]
	\addplot[blue,semithick] file {wykresy/pid_3_2_4/6.6821_4.5492_2.6147_3.0000_5.0000_3.0000_0.0000_0.0000_0.0000/u3.txt};
	\addplot[green,semithick] file {wykresy/pid_3_2_4/7.8600_4.2845_1.6460_2.1371_4.3444_1.8913_0.0000_0.0132_0.0000/u3.txt};
	\legend{$Nasze$,$Obliczone$}
	\nextgroupplot
	[
	xmin=0,xmax=910,ymin=-50,ymax=50,
	xlabel={$k$},
	ylabel={$u2$},
	xtick={0,100,200,300,400,500,600,700,800,900},
	ytick={-50,-25,0,25,50},
	y tick label style={/pgf/number format/1000 sep=},
	]
	\addplot[blue,semithick] file {wykresy/pid_3_2_4/6.6821_4.5492_2.6147_3.0000_5.0000_3.0000_0.0000_0.0000_0.0000/u2.txt};
	\addplot[green,semithick] file {wykresy/pid_3_2_4/7.8600_4.2845_1.6460_2.1371_4.3444_1.8913_0.0000_0.0132_0.0000/u2.txt};
	\legend{$Nasze$,$Obliczone$}
	\nextgroupplot
	[
	xmin=0,xmax=910,ymin=-50,ymax=50,
	xlabel={$k$},
	ylabel={$u4$},
	xtick={0,100,200,300,400,500,600,700,800,900},
	ytick={-50,-25,0,25,50},
	y tick label style={/pgf/number format/1000 sep=},
	]
	\addplot[blue,semithick] file {wykresy/pid_3_2_4/6.6821_4.5492_2.6147_3.0000_5.0000_3.0000_0.0000_0.0000_0.0000/u4.txt};
	\addplot[green,semithick] file {wykresy/pid_3_2_4/7.8600_4.2845_1.6460_2.1371_4.3444_1.8913_0.0000_0.0132_0.0000/u4.txt};
	\legend{$Nasze$,$Obliczone$}
	\end{groupplot}
	\end{tikzpicture}
	\caption{Por�wnanie wyliczonych i optymalnych nastaw dla konfiguracji pierwszej - sterowanie}
	\label{pid1s}
\end{figure}

\begin{figure}[tb]
	\centering
	\begin{tikzpicture}
	\begin{groupplot}[group style={group size=1 by 3,vertical sep={1.5 cm}},
	width=0.9\textwidth,height=0.4\textwidth]
	\nextgroupplot
	[
	xmin=0,xmax=910,ymin=-10,ymax=10,
	xlabel={$k$},
	ylabel={$y1$},
	xtick={0,100,200,300,400,500,600,700,800,900},
	ytick={-10,-5,0,5,10},
	y tick label style={/pgf/number format/1000 sep=},
	legend pos=south east,
	]
	\addplot[blue,semithick] file {wykresy/pid_3_2_4/6.6821_4.5492_2.6147_3.0000_5.0000_3.0000_0.0000_0.0000_0.0000/y1.txt};
	\addplot[green,semithick] file {wykresy/pid_3_2_4/7.8600_4.2845_1.6460_2.1371_4.3444_1.8913_0.0000_0.0132_0.0000/y1.txt};
	\addplot[red,semithick,densely dashed] file {wykresy/pid_3_2_4/6.6821_4.5492_2.6147_3.0000_5.0000_3.0000_0.0000_0.0000_0.0000/yzad1.txt};
	\legend{$Nasze$,$Obliczone$,$y1_{zad}$}
	\nextgroupplot
	[
	xmin=0,xmax=910,ymin=-10,ymax=10,
	xlabel={$k$},
	ylabel={$y2$},
	xtick={0,100,200,300,400,500,600,700,800,900},
	ytick={-10,-5,0,5,10},
	y tick label style={/pgf/number format/1000 sep=},
	legend pos=south east,
	]
	\addplot[blue,semithick] file {wykresy/pid_3_2_4/6.6821_4.5492_2.6147_3.0000_5.0000_3.0000_0.0000_0.0000_0.0000/y2.txt};
	\addplot[green,semithick] file {wykresy/pid_3_2_4/7.8600_4.2845_1.6460_2.1371_4.3444_1.8913_0.0000_0.0132_0.0000/y2.txt};
	\addplot[red,semithick,densely dashed] file {wykresy/pid_3_2_4/6.6821_4.5492_2.6147_3.0000_5.0000_3.0000_0.0000_0.0000_0.0000/yzad2.txt};
	\legend{$Nasze$,$Obliczone$,$y2_{zad}$}
	\nextgroupplot
	[
	xmin=0,xmax=910,ymin=-10,ymax=10,
	xlabel={$k$},
	ylabel={$y3$},
	xtick={0,100,200,300,400,500,600,700,800,900},
	ytick={-10,-5,0,5,10},
	y tick label style={/pgf/number format/1000 sep=},
	legend pos=south east,
	]
	\addplot[blue,semithick] file {wykresy/pid_3_2_4/6.6821_4.5492_2.6147_3.0000_5.0000_3.0000_0.0000_0.0000_0.0000/y3.txt};
	\addplot[green,semithick] file {wykresy/pid_3_2_4/7.8600_4.2845_1.6460_2.1371_4.3444_1.8913_0.0000_0.0132_0.0000/y3.txt};
	\addplot[red,semithick,densely dashed] file {wykresy/pid_3_2_4/6.6821_4.5492_2.6147_3.0000_5.0000_3.0000_0.0000_0.0000_0.0000/yzad3.txt};
	\legend{$Nasze$,$Obliczone$,$y3_{zad}$}
	\end{groupplot}
	\end{tikzpicture}
	\caption{Por�wnanie wyliczonych i optymalnych nastaw dla konfiguracji pierwszej - wyj�cia}
	\label{pid1w}
\end{figure}
\FloatBarrier
\subsection{Konfiguracja druga}
Parametry obliczone przez program nie s� bardzo oddalone od naszych, ale gwarantuj� mniejsz� warto�� b��du. Cho� przebiegi sterowa� s� prawie identyczne, na wykresach wyj�� w definitywnie wida� popraw�.

\begin{table}
	\centering
	\begin{tabular}{|c|c|c|c|}
		\hline
		& $\boldsymbol{y1}$ & $\boldsymbol{y2}$ & $\boldsymbol{y3}$ \\ \hline
		$\boldsymbol{K}$ & 6,6821 & 2,6806 & 1,7432 \\ \hline
		$\boldsymbol{T_i}$ & 3 & 2 & 1 \\ \hline
		$\boldsymbol{T_d}$ & 0 & 0 & 0 \\ \hline
	\end{tabular}
\caption{Nasze nastawy}
	\label{pid2n1}

	\centering
	\begin{tabular}{c|c|c|c}
		$\boldsymbol{E}$ & $\boldsymbol{E1}$ & $\boldsymbol{E2}$ & $\boldsymbol{E3}$ \\ \hline
		84,5872 & 43,0392 & 14,1660 & 27,3819 \\
	\end{tabular}
\caption{Warto�ci b��d�w dla naszych nastaw}
	\label{pid2e1}
\end{table}

\begin{table}
	\centering
	\begin{tabular}{|c|c|c|c|}
		\hline
		& $\boldsymbol{y1}$ & $\boldsymbol{y2}$ & $\boldsymbol{y3}$ \\ \hline
		$\boldsymbol{K}$ & 8,0752 & 2,3354 & 1,5439 \\ \hline
		$\boldsymbol{T_i}$ & 1,8399 & 2,3517 & 1,4822 \\ \hline
		$\boldsymbol{T_d}$ & 4,3152e-9 & 0,0086 & 8,0777e-7 \\ \hline
	\end{tabular}
	\caption{Wyliczone nastawy}
	\label{pid2n2}
\end{table}

\begin{table}
	\centering
	\begin{tabular}{c|c|c|c}
		$\boldsymbol{E}$ & $\boldsymbol{E1}$ & $\boldsymbol{E2}$ & $\boldsymbol{E3}$ \\ \hline
		74,8601 & 28,8221 & 14,6279 & 31,4101 \\
	\end{tabular}
	\caption{Warto�ci b��d�w dla wyliczonych nastaw}
	\label{pid2e2}
\end{table}

\begin{figure}[tb]
	\centering
	\begin{tikzpicture}
	\begin{groupplot}[group style={group size=1 by 3,vertical sep={1.5 cm}},
	width=0.9\textwidth,height=0.4\textwidth]
	\nextgroupplot
	[
	xmin=0,xmax=910,ymin=-50,ymax=50,
	xlabel={$k$},
	ylabel={$u3$},
	xtick={0,100,200,300,400,500,600,700,800,900},
	ytick={-50,-25,0,25,50},
	y tick label style={/pgf/number format/1000 sep=},
	]
	\addplot[blue,semithick] file {wykresy/pid_3_1_4/6.6821_2.6806_1.7432_3.0000_2.0000_1.0000_0.0000_0.0000_0.0000/u3.txt};
	\addplot[green,semithick] file {wykresy/pid_3_1_4/8.0752_2.3354_1.5439_1.8399_2.3517_1.4822_0.0000_0.0086_0.0000/u3.txt};
	\legend{$Nasze$,$Obliczone$}
	\nextgroupplot
	[
	xmin=0,xmax=910,ymin=-50,ymax=50,
	xlabel={$k$},
	ylabel={$u1$},
	xtick={0,100,200,300,400,500,600,700,800,900},
	ytick={-50,-25,0,25,50},
	y tick label style={/pgf/number format/1000 sep=},
	]
	\addplot[blue,semithick] file {wykresy/pid_3_1_4/6.6821_2.6806_1.7432_3.0000_2.0000_1.0000_0.0000_0.0000_0.0000/u1.txt};
	\addplot[green,semithick] file {wykresy/pid_3_1_4/8.0752_2.3354_1.5439_1.8399_2.3517_1.4822_0.0000_0.0086_0.0000/u1.txt};
	\legend{$Nasze$,$Obliczone$}
	\nextgroupplot
	[
	xmin=0,xmax=910,ymin=-50,ymax=50,
	xlabel={$k$},
	ylabel={$u4$},
	xtick={0,100,200,300,400,500,600,700,800,900},
	ytick={-50,-25,0,25,50},
	y tick label style={/pgf/number format/1000 sep=},
	]
	\addplot[blue,semithick] file {wykresy/pid_3_1_4/6.6821_2.6806_1.7432_3.0000_2.0000_1.0000_0.0000_0.0000_0.0000/u4.txt};
	\addplot[green,semithick] file {wykresy/pid_3_1_4/8.0752_2.3354_1.5439_1.8399_2.3517_1.4822_0.0000_0.0086_0.0000/u4.txt};
	\legend{$Nasze$,$Obliczone$}
	
	\end{groupplot}
	\end{tikzpicture}
	\caption{Por�wnanie wyliczonych i optymalnych nastaw dla konfiguracji drugiej - sterowanie}
	\label{pid2s}
\end{figure}

\begin{figure}[tb]
	\centering
	\begin{tikzpicture}
	\begin{groupplot}[group style={group size=1 by 3,vertical sep={1.5 cm}},
	width=0.9\textwidth,height=0.4\textwidth]
	\nextgroupplot
	[
	xmin=0,xmax=910,ymin=-10,ymax=10,
	xlabel={$k$},
	ylabel={$y1$},
	xtick={0,100,200,300,400,500,600,700,800,900},
	ytick={-10,-5,0,5,10},
	y tick label style={/pgf/number format/1000 sep=},
	legend pos=south east,
	]
	\addplot[blue,semithick] file {wykresy/pid_3_1_4/6.6821_2.6806_1.7432_3.0000_2.0000_1.0000_0.0000_0.0000_0.0000/y1.txt};
	\addplot[green,semithick] file {wykresy/pid_3_1_4/8.0752_2.3354_1.5439_1.8399_2.3517_1.4822_0.0000_0.0086_0.0000/y1.txt};
	\addplot[red,semithick,densely dashed] file {wykresy/pid_3_1_4/6.6821_2.6806_1.7432_3.0000_2.0000_1.0000_0.0000_0.0000_0.0000/yzad1.txt};
	\legend{$Nasze$,$Obliczone$,$y1_{zad}$}
	\nextgroupplot
	[
	xmin=0,xmax=910,ymin=-10,ymax=10,
	xlabel={$k$},
	ylabel={$y2$},
	xtick={0,100,200,300,400,500,600,700,800,900},
	ytick={-10,-5,0,5,10},
	y tick label style={/pgf/number format/1000 sep=},
	legend pos=south east,
	]
	\addplot[blue,semithick] file {wykresy/pid_3_1_4/6.6821_2.6806_1.7432_3.0000_2.0000_1.0000_0.0000_0.0000_0.0000/y2.txt};
	\addplot[green,semithick] file {wykresy/pid_3_1_4/8.0752_2.3354_1.5439_1.8399_2.3517_1.4822_0.0000_0.0086_0.0000/y2.txt};
	\addplot[red,semithick,densely dashed] file {wykresy/pid_3_1_4/6.6821_2.6806_1.7432_3.0000_2.0000_1.0000_0.0000_0.0000_0.0000/yzad2.txt};
	\legend{$Nasze$,$Obliczone$,$y2_{zad}$}
	\nextgroupplot
	[
	xmin=0,xmax=910,ymin=-10,ymax=10,
	xlabel={$k$},
	ylabel={$y3$},
	xtick={0,100,200,300,400,500,600,700,800,900},
	ytick={-10,-5,0,5,10},
	y tick label style={/pgf/number format/1000 sep=},
	legend pos=south east,
	]
	\addplot[blue,semithick] file {wykresy/pid_3_1_4/6.6821_2.6806_1.7432_3.0000_2.0000_1.0000_0.0000_0.0000_0.0000/y3.txt};
	\addplot[green,semithick] file {wykresy/pid_3_1_4/8.0752_2.3354_1.5439_1.8399_2.3517_1.4822_0.0000_0.0086_0.0000/y3.txt};
	\addplot[red,semithick,densely dashed] file {wykresy/pid_3_1_4/6.6821_2.6806_1.7432_3.0000_2.0000_1.0000_0.0000_0.0000_0.0000/yzad3.txt};
	\legend{$Nasze$,$Obliczone$,$y3_{zad}$}
	\end{groupplot}
	\end{tikzpicture}
	\caption{Por�wnanie wyliczonych i optymalnych nastaw dla konfiguracji drugiej - wyj�cia}
	\label{pid2w}
\end{figure}
\FloatBarrier
\subsection{Konfiguracja trzecia}
Przy nastawach wyliczonych za pomoc� optymalizacji poprawi�y si� zar�wno b��dy jak i przebiegi wykres�w wyj��. Mimo wszystko warto�ci b��d�w wci�� s� do�� du�e w por�wnaniu do dw�ch pierwszych konfiguracji. Wzbuczi�o to w nas pewne zdumienie, bo to w�a�nie ta konfiguracja by�a wskazywana przez wska�nik uwarunkowania jako potentat do bycia najlepsz� ze wszystkich.

\begin{table}
	\centering
	\begin{tabular}{|c|c|c|c|}
		\hline
& $\boldsymbol{y1}$ & $\boldsymbol{y2}$ & $\boldsymbol{y3}$ \\ \hline
$\boldsymbol{K}$ & 1,1337 & 2,6806 & 0,5811 \\ \hline
$\boldsymbol{T_i}$ & 6 & 1 & 4 \\ \hline
$\boldsymbol{T_d}$ & 2 & 0 & 0 \\ \hline
	\end{tabular}
	\caption{Nasze nastawy}
	\label{pid3n1}
	
	\centering
	\begin{tabular}{c|c|c|c}
		$\boldsymbol{E}$ & $\boldsymbol{E1}$ & $\boldsymbol{E2}$ & $\boldsymbol{E3}$ \\ \hline
		295,8916 & 139,8703 & 21,7816 & 134,2398 \\
	\end{tabular}
	\caption{Warto�ci b��d�w dla naszych nastaw}
	\label{pid3e1}
\end{table}

\begin{table}
	\centering
	\begin{tabular}{|c|c|c|c|}
		\hline
		& $\boldsymbol{y1}$ & $\boldsymbol{y2}$ & $\boldsymbol{y3}$ \\ \hline
		$\boldsymbol{K}$ & 5,0737 & 2,9191 & 0,7908 \\ \hline
		$\boldsymbol{T_i}$ & 16,1926 & 0,5115 & 3,9598 \\ \hline
		$\boldsymbol{T_d}$ & 3,8791e-6 & 6,1826e-6 & 0,4474 \\ \hline
	\end{tabular}
	\caption{Wyliczone nastawy}
	\label{pid3n2}
\end{table}

\begin{table}
	\centering
	\begin{tabular}{c|c|c|c}
		$\boldsymbol{E}$ & $\boldsymbol{E1}$ & $\boldsymbol{E2}$ & $\boldsymbol{E3}$ \\ \hline
		205,9376 & 63,2208 & 26,6137 & 116,1032 \\
	\end{tabular}
	\caption{Warto�ci b��d�w dla wyliczonych nastaw}
	\label{pid3e2}
\end{table}

\begin{figure}[tb]
	\centering
	\begin{tikzpicture}
	\begin{groupplot}[group style={group size=1 by 3,vertical sep={1.5 cm}},
	width=0.9\textwidth,height=0.4\textwidth]
	\nextgroupplot
	[
	xmin=0,xmax=910,ymin=-50,ymax=50,
	xlabel={$k$},
	ylabel={$u2$},
	xtick={0,100,200,300,400,500,600,700,800,900},
	ytick={-50,-25,0,25,50},
	y tick label style={/pgf/number format/1000 sep=},
	]
	\addplot[blue,semithick] file {wykresy/pid_2_1_4/1.1337_2.6806_0.5811_6.0000_1.0000_4.0000_2.0000_0.0000_0.0000/u2.txt};
	\addplot[green,semithick] file {wykresy/pid_2_1_4/5.0737_2.9191_0.7908_16.1926_0.5115_3.9598_0.0000_0.0000_0.4474/u2.txt};
	\legend{$Nasze$,$Obliczone$}
	\nextgroupplot
	[
	xmin=0,xmax=910,ymin=-50,ymax=50,
	xlabel={$k$},
	ylabel={$u1$},
	xtick={0,100,200,300,400,500,600,700,800,900},
	ytick={-50,-25,0,25,50},
	y tick label style={/pgf/number format/1000 sep=},
	]
	\addplot[blue,semithick] file {wykresy/pid_2_1_4/1.1337_2.6806_0.5811_6.0000_1.0000_4.0000_2.0000_0.0000_0.0000/u1.txt};
	\addplot[green,semithick] file {wykresy/pid_2_1_4/5.0737_2.9191_0.7908_16.1926_0.5115_3.9598_0.0000_0.0000_0.4474/u1.txt};
	\legend{$Nasze$,$Obliczone$}
	\nextgroupplot
	[
	xmin=0,xmax=910,ymin=-50,ymax=50,
	xlabel={$k$},
	ylabel={$u4$},
	xtick={0,100,200,300,400,500,600,700,800,900},
	ytick={-50,-25,0,25,50},
	y tick label style={/pgf/number format/1000 sep=},
	]
	\addplot[blue,semithick] file {wykresy/pid_2_1_4/1.1337_2.6806_0.5811_6.0000_1.0000_4.0000_2.0000_0.0000_0.0000/u4.txt};
	\addplot[green,semithick] file {wykresy/pid_2_1_4/5.0737_2.9191_0.7908_16.1926_0.5115_3.9598_0.0000_0.0000_0.4474/u4.txt};
	\legend{$Nasze$,$Obliczone$}
	
	\end{groupplot}
	\end{tikzpicture}
	\caption{Por�wnanie wyliczonych i optymalnych nastaw dla konfiguracji trzeciej - sterowanie}
	\label{pid3s}
\end{figure}

\begin{figure}[tb]
	\centering
	\begin{tikzpicture}
	\begin{groupplot}[group style={group size=1 by 3,vertical sep={1.5 cm}},
	width=0.9\textwidth,height=0.4\textwidth]
	\nextgroupplot
	[
	xmin=0,xmax=910,ymin=-10,ymax=10,
	xlabel={$k$},
	ylabel={$y1$},
	xtick={0,100,200,300,400,500,600,700,800,900},
	ytick={-10,-5,0,5,10},
	y tick label style={/pgf/number format/1000 sep=},
	legend pos=south east,
	]
	\addplot[blue,semithick] file {wykresy/pid_2_1_4/1.1337_2.6806_0.5811_6.0000_1.0000_4.0000_2.0000_0.0000_0.0000/y1.txt};
	\addplot[green,semithick] file {wykresy/pid_2_1_4/5.0737_2.9191_0.7908_16.1926_0.5115_3.9598_0.0000_0.0000_0.4474/y1.txt};
	\addplot[red,semithick,densely dashed] file {wykresy/pid_2_1_4/5.0737_2.9191_0.7908_16.1926_0.5115_3.9598_0.0000_0.0000_0.4474/yzad1.txt};
	\legend{$Nasze$,$Obliczone$,$y1_{zad}$}
	\nextgroupplot
	[
	xmin=0,xmax=910,ymin=-10,ymax=10,
	xlabel={$k$},
	ylabel={$y2$},
	xtick={0,100,200,300,400,500,600,700,800,900},
	ytick={-10,-5,0,5,10},
	y tick label style={/pgf/number format/1000 sep=},
	legend pos=south east,
	]
	\addplot[blue,semithick] file {wykresy/pid_2_1_4/1.1337_2.6806_0.5811_6.0000_1.0000_4.0000_2.0000_0.0000_0.0000/y2.txt};
	\addplot[green,semithick] file {wykresy/pid_2_1_4/5.0737_2.9191_0.7908_16.1926_0.5115_3.9598_0.0000_0.0000_0.4474/y2.txt};
	\addplot[red,semithick,densely dashed] file {wykresy/pid_2_1_4/5.0737_2.9191_0.7908_16.1926_0.5115_3.9598_0.0000_0.0000_0.4474/yzad2.txt};
	\legend{$Nasze$,$Obliczone$,$y2_{zad}$}
	\nextgroupplot
	[
	xmin=0,xmax=910,ymin=-10,ymax=10,
	xlabel={$k$},
	ylabel={$y3$},
	xtick={0,100,200,300,400,500,600,700,800,900},
	ytick={-10,-5,0,5,10},
	y tick label style={/pgf/number format/1000 sep=},
	legend pos=south east,
	]
	\addplot[blue,semithick] file {wykresy/pid_2_1_4/1.1337_2.6806_0.5811_6.0000_1.0000_4.0000_2.0000_0.0000_0.0000/y3.txt};
	\addplot[green,semithick] file {wykresy/pid_2_1_4/5.0737_2.9191_0.7908_16.1926_0.5115_3.9598_0.0000_0.0000_0.4474/y3.txt};
	\addplot[red,semithick,densely dashed] file {wykresy/pid_2_1_4/5.0737_2.9191_0.7908_16.1926_0.5115_3.9598_0.0000_0.0000_0.4474/yzad3.txt};
	\legend{$Nasze$,$Obliczone$,$y3_{zad}$}
	\end{groupplot}
	\end{tikzpicture}
	\caption{Por�wnanie wyliczonych i optymalnych nastaw dla konfiguracji trzeciej - wyj�cia}
	\label{pid3w}
\end{figure}
\FloatBarrier
\subsection{Konfiguracja czwarta}
Wyliczenie nastaw za pomoc� optymalizacji nie da�o du�ej poprawy zar�wno w warto�ciach b��d�w jak i w wygl�dzie przebieg�w. Najwyra�niej akurat ta konfiguracja nie jest najlepszym wyborem je�li chodzi o sterowanie posiadanym obiektem.

\begin{table}
	\centering
	\begin{tabular}{|c|c|c|c|}
		\hline
& $\boldsymbol{y1}$ & $\boldsymbol{y2}$ & $\boldsymbol{y3}$ \\ \hline
$\boldsymbol{K}$ & 6,6821 & 3,0328 & 0,7699 \\ \hline
$\boldsymbol{T_i}$ & 1 & 7 & 5 \\ \hline
$\boldsymbol{T_d}$ & 0 & 0 & 1 \\ \hline
	\end{tabular}
	\caption{Nasze nastawy}
	\label{pid4n1}
	
	\centering
	\begin{tabular}{c|c|c|c}
		$\boldsymbol{E}$ & $\boldsymbol{E1}$ & $\boldsymbol{E2}$ & $\boldsymbol{E3}$ \\ \hline
279,8519 & 17,2126 & 84,7003 & 177,9390 \\
	\end{tabular}
	\caption{Warto�ci b��d�w dla naszych nastaw}
	\label{pid4e1}
\end{table}

\begin{table}
	\centering
	\begin{tabular}{|c|c|c|c|}
		\hline
		& $\boldsymbol{y1}$ & $\boldsymbol{y2}$ & $\boldsymbol{y3}$ \\ \hline
		$\boldsymbol{K}$ & 6,9911 & 5,2839 & 1,1192 \\ \hline
		$\boldsymbol{T_i}$ & 1,0949 & 12,0669 & 0,5076 \\ \hline
		$\boldsymbol{T_d}$ & 8,8144e-5 & 3,1067e-6 & 0,5076 \\ \hline
	\end{tabular}
	\caption{Wyliczone nastawy}
	\label{pid4n2}
\end{table}

\begin{table}
	\centering
	\begin{tabular}{c|c|c|c}
		$\boldsymbol{E}$ & $\boldsymbol{E1}$ & $\boldsymbol{E2}$ & $\boldsymbol{E3}$ \\ \hline
		259,3231 & 19,3182 & 65,3477 & 174,6572 \\
	\end{tabular}
	\caption{Warto�ci b��d�w dla wyliczonych nastaw}
	\label{pid4e2}
\end{table}

\begin{figure}[tb]
	\centering
	\begin{tikzpicture}
	\begin{groupplot}[group style={group size=1 by 3,vertical sep={1.5 cm}},
	width=0.9\textwidth,height=0.4\textwidth]
	\nextgroupplot
	[
	xmin=0,xmax=910,ymin=-50,ymax=50,
	xlabel={$k$},
	ylabel={$u2$},
	xtick={0,100,200,300,400,500,600,700,800,900},
	ytick={-50,-25,0,25,50},
	y tick label style={/pgf/number format/1000 sep=},
	]
	\addplot[blue,semithick] file {wykresy/pid_3_2_1/6.6821_3.0328_0.7699_1.0000_7.0000_5.0000_0.0000_0.0000_1.0000/u3.txt};
	\addplot[green,semithick] file {wykresy/pid_3_2_1/6.9911_5.2839_1.1192_1.0949_12.0669_7.0111_0.0001_0.0000_0.5076/u3.txt};
	\legend{$Nasze$,$Obliczone$}
	\nextgroupplot
	[
	xmin=0,xmax=910,ymin=-50,ymax=50,
	xlabel={$k$},
	ylabel={$u1$},
	xtick={0,100,200,300,400,500,600,700,800,900},
	ytick={-50,-25,0,25,50},
	y tick label style={/pgf/number format/1000 sep=},
	]
	\addplot[blue,semithick] file {wykresy/pid_3_2_1/6.6821_3.0328_0.7699_1.0000_7.0000_5.0000_0.0000_0.0000_1.0000/u2.txt};
	\addplot[green,semithick] file {wykresy/pid_3_2_1/6.9911_5.2839_1.1192_1.0949_12.0669_7.0111_0.0001_0.0000_0.5076/u2.txt};
	\legend{$Nasze$,$Obliczone$}
	\nextgroupplot
	[
	xmin=0,xmax=910,ymin=-50,ymax=50,
	xlabel={$k$},
	ylabel={$u4$},
	xtick={0,100,200,300,400,500,600,700,800,900},
	ytick={-50,-25,0,25,50},
	y tick label style={/pgf/number format/1000 sep=},
	]
	\addplot[blue,semithick] file {wykresy/pid_3_2_1/6.6821_3.0328_0.7699_1.0000_7.0000_5.0000_0.0000_0.0000_1.0000/u1.txt};
	\addplot[green,semithick] file {wykresy/pid_3_2_1/6.9911_5.2839_1.1192_1.0949_12.0669_7.0111_0.0001_0.0000_0.5076/u1.txt};
	\legend{$Nasze$,$Obliczone$}
	
	\end{groupplot}
	\end{tikzpicture}
	\caption{Por�wnanie wyliczonych i optymalnych nastaw dla konfiguracji czwartej - sterowanie}
	\label{pid4s}
\end{figure}

\begin{figure}[tb]
	\centering
	\begin{tikzpicture}
	\begin{groupplot}[group style={group size=1 by 3,vertical sep={1.5 cm}},
	width=0.9\textwidth,height=0.4\textwidth]
	\nextgroupplot
	[
	xmin=0,xmax=910,ymin=-10,ymax=10,
	xlabel={$k$},
	ylabel={$y1$},
	xtick={0,100,200,300,400,500,600,700,800,900},
	ytick={-10,-5,0,5,10},
	y tick label style={/pgf/number format/1000 sep=},
	legend pos=south east,
	]
	\addplot[blue,semithick] file {wykresy/pid_3_2_1/6.6821_3.0328_0.7699_1.0000_7.0000_5.0000_0.0000_0.0000_1.0000/y1.txt};
	\addplot[green,semithick] file {wykresy/pid_3_2_1/6.9911_5.2839_1.1192_1.0949_12.0669_7.0111_0.0001_0.0000_0.5076/y1.txt};
	\addplot[red,semithick,densely dashed] file {wykresy/pid_2_1_4/5.0737_2.9191_0.7908_16.1926_0.5115_3.9598_0.0000_0.0000_0.4474/yzad1.txt};
	\legend{$Nasze$,$Obliczone$,$y1_{zad}$}
	\nextgroupplot
	[
	xmin=0,xmax=910,ymin=-10,ymax=10,
	xlabel={$k$},
	ylabel={$y2$},
	xtick={0,100,200,300,400,500,600,700,800,900},
	ytick={-10,-5,0,5,10},
	y tick label style={/pgf/number format/1000 sep=},
	legend pos=south east,
	]
	\addplot[blue,semithick] file {wykresy/pid_3_2_1/6.6821_3.0328_0.7699_1.0000_7.0000_5.0000_0.0000_0.0000_1.0000/y2.txt};
	\addplot[green,semithick] file {wykresy/pid_3_2_1/6.9911_5.2839_1.1192_1.0949_12.0669_7.0111_0.0001_0.0000_0.5076/y2.txt};
	\addplot[red,semithick,densely dashed] file {wykresy/pid_2_1_4/5.0737_2.9191_0.7908_16.1926_0.5115_3.9598_0.0000_0.0000_0.4474/yzad2.txt};
	\legend{$Nasze$,$Obliczone$,$y2_{zad}$}
	\nextgroupplot
	[
	xmin=0,xmax=910,ymin=-10,ymax=10,
	xlabel={$k$},
	ylabel={$y3$},
	xtick={0,100,200,300,400,500,600,700,800,900},
	ytick={-10,-5,0,5,10},
	y tick label style={/pgf/number format/1000 sep=},
	legend pos=south east,
	]
	\addplot[blue,semithick] file {wykresy/pid_3_2_1/6.6821_3.0328_0.7699_1.0000_7.0000_5.0000_0.0000_0.0000_1.0000/y3.txt};
	\addplot[green,semithick] file {wykresy/pid_3_2_1/6.9911_5.2839_1.1192_1.0949_12.0669_7.0111_0.0001_0.0000_0.5076/y3.txt};
	\addplot[red,semithick,densely dashed] file {wykresy/pid_2_1_4/5.0737_2.9191_0.7908_16.1926_0.5115_3.9598_0.0000_0.0000_0.4474/yzad3.txt};
	\legend{$Nasze$,$Obliczone$,$y3_{zad}$}
	\end{groupplot}
	\end{tikzpicture}
	\caption{Por�wnanie wyliczonych i optymalnych nastaw dla konfiguracji czwartej - wyj�cia}
	\label{pid4w}
\end{figure}
\FloatBarrier
\section{DMC}
Ostatni� optymalizacj� jak� nale�y przeprowadzi� jest optymalizacja DMC. Wi�kszo�� nastaw obliczonych w wyniku optymalizacji jest podobna do wyliczonych przez nas. Mimo wszystko optymalizacja zmniejszy�a ju� ma�e b��dy przebieg�w i definitywnie poprawi�a wykresy wyj��.

\begin{table}
	\centering
	\begin{tabular}{|c|c|c|c|c|c|c|}
		\hline
		$\boldsymbol{\psi_1}$ & $\boldsymbol{\psi_2}$ & $\boldsymbol{\psi_3}$ & $\boldsymbol{\lambda_1}$ & $\boldsymbol{\lambda_2}$ & $\boldsymbol{\lambda_3}$ & $\boldsymbol{\lambda_4}$ \\ \hline
2,7 & 40 & 7 & 20 & 1 & 0 & 0 \\ \hline
	\end{tabular}
	\caption{Nasze nastawy}
	\label{dmcn1}
	
	\centering
	\begin{tabular}{c|c|c|c}
		$\boldsymbol{E}$ & $\boldsymbol{E1}$ & $\boldsymbol{E2}$ & $\boldsymbol{E3}$ \\ \hline
		49,5733 & 16,8105 & 18,6579 & 14,1050 \\
	\end{tabular}
	\caption{Warto�ci b��d�w dla naszych nastaw}
	\label{dmce1}
\end{table}

\begin{table}
	\centering
	\begin{tabular}{|c|c|c|c|c|c|c|}
		\hline
$\boldsymbol{\psi_1}$ & $\boldsymbol{\psi_2}$ & $\boldsymbol{\psi_3}$ & $\boldsymbol{\lambda_1}$ & $\boldsymbol{\lambda_2}$ & $\boldsymbol{\lambda_3}$ & $\boldsymbol{\lambda_4}$ \\ \hline
6,7137 & 40,1488 & 7,3852 & 20,0812 & 0,0007 & 2,1222e-5 & 6,7126e-5 \\ \hline
	\end{tabular}
	\caption{Wyliczone nastawy}
	\label{dmcn2}
\end{table}

\begin{table}
	\centering
	\begin{tabular}{c|c|c|c}
		$\boldsymbol{E}$ & $\boldsymbol{E1}$ & $\boldsymbol{E2}$ & $\boldsymbol{E3}$ \\ \hline
		42,0328 & 14,0032 & 14,0061 & 14,0235 \\
	\end{tabular}
	\caption{Warto�ci b��d�w dla wyliczonych nastaw}
	\label{dmce2}
\end{table}

\begin{figure}[tb]
	\centering
	\begin{tikzpicture}
	\begin{groupplot}[group style={group size=1 by 4,vertical sep={1 cm}},
	width=0.9\textwidth,height=0.4\textwidth]
	\nextgroupplot
	[
	xmin=0,xmax=910,ymin=-10,ymax=10,
	xlabel={$k$},
	ylabel={$u1$},
	xtick={0,100,200,300,400,100,600,700,800,900},
	ytick={-10,-5,0,5,10},
	y tick label style={/pgf/number format/1000 sep=},
	legend pos=south east,
	]
	\addplot[blue,semithick] file {wykresy/dmc_200_200_200/2.7000_40.0000_7.0000_20.0000_1.0000_0.0000_0.0000/u1.txt};
	\addplot[green,semithick] file {wykresy/dmc_200_200_200/6.7137_40.1488_7.3852_20.0812_0.0007_0.0000_0.0001/u1.txt};
	\legend{$Nasze$,$Obliczone$}
	\nextgroupplot
	[
	xmin=0,xmax=910,ymin=-10,ymax=10,
	xlabel={$k$},
	ylabel={$u2$},
	xtick={0,100,200,300,400,100,600,700,800,900},
	ytick={-10,-5,0,5,10},
	y tick label style={/pgf/number format/1000 sep=},
	legend pos=south east,
	]
	\addplot[blue,semithick] file {wykresy/dmc_200_200_200/2.7000_40.0000_7.0000_20.0000_1.0000_0.0000_0.0000/u2.txt};
	\addplot[green,semithick] file {wykresy/dmc_200_200_200/6.7137_40.1488_7.3852_20.0812_0.0007_0.0000_0.0001/u2.txt};
	\legend{$Nasze$,$Obliczone$}
	\nextgroupplot
	[
	xmin=0,xmax=910,ymin=-10,ymax=10,
	xlabel={$k$},
	ylabel={$u3$},
	xtick={0,100,200,300,400,100,600,700,800,900},
	ytick={-10,-5,0,5,10},
	y tick label style={/pgf/number format/1000 sep=},
	legend pos=south east,
	]
	\addplot[blue,semithick] file {wykresy/dmc_200_200_200/2.7000_40.0000_7.0000_20.0000_1.0000_0.0000_0.0000/u3.txt};
	\addplot[green,semithick] file {wykresy/dmc_200_200_200/6.7137_40.1488_7.3852_20.0812_0.0007_0.0000_0.0001/u3.txt};
	\legend{$Nasze$,$Obliczone$}
	\nextgroupplot
	[
	xmin=0,xmax=910,ymin=-10,ymax=10,
	xlabel={$k$},
	ylabel={$u4$},
	xtick={0,100,200,300,400,100,600,700,800,900},
	ytick={-10,-5,0,5,10},
	y tick label style={/pgf/number format/1000 sep=},
	legend pos=north east,
	]
	\addplot[blue,semithick] file {wykresy/dmc_200_200_200/2.7000_40.0000_7.0000_20.0000_1.0000_0.0000_0.0000/u4.txt};
	\addplot[green,semithick] file {wykresy/dmc_200_200_200/6.7137_40.1488_7.3852_20.0812_0.0007_0.0000_0.0001/u4.txt};
	\legend{$Nasze$,$Obliczone$}
	\end{groupplot}
	\end{tikzpicture}
	\caption{Por�wnanie wyliczonych i optymalnych nastaw dla DMC - sterowanie}
	\label{DMCs}
\end{figure}

\begin{figure}[tb]
	\centering
	\begin{tikzpicture}
	\begin{groupplot}[group style={group size=1 by 3,vertical sep={1 cm}},
	width=0.9\textwidth,height=0.4\textwidth]
	\nextgroupplot
	[
	xmin=0,xmax=910,ymin=-5,ymax=5,
	xlabel={$k$},
	ylabel={$y1$},
	xtick={0,100,200,300,400,100,600,700,800,900},
	ytick={-5,-2.5,0,2.5,5},
	y tick label style={/pgf/number format/1000 sep=},
	legend pos=south east,
	]
	\addplot[blue,semithick] file {wykresy/dmc_200_200_200/2.7000_40.0000_7.0000_20.0000_1.0000_0.0000_0.0000/y1.txt};
	\addplot[green,semithick] file {wykresy/dmc_200_200_200/6.7137_40.1488_7.3852_20.0812_0.0007_0.0000_0.0001/y1.txt};
	\addplot[red,semithick,densely dashed] file {wykresy/dmc_200_200_200/1.0000_1.0000_1.0000_1.0000_1.0000_1.0000_1.0000/yzad1.txt};
	\legend{$Nasze$,$Obliczone$,$y1_{zad}$}
	\nextgroupplot
	[
	xmin=0,xmax=910,ymin=-5,ymax=5,
	xlabel={$k$},
	ylabel={$y2$},
	xtick={0,100,200,300,400,100,600,700,800,900},
	ytick={-5,-2.5,0,2.5,5},
	y tick label style={/pgf/number format/1000 sep=},
	legend pos=south east,
	] 
	\addplot[blue,semithick] file {wykresy/dmc_200_200_200/2.7000_40.0000_7.0000_20.0000_1.0000_0.0000_0.0000/y2.txt};
	\addplot[green,semithick] file {wykresy/dmc_200_200_200/6.7137_40.1488_7.3852_20.0812_0.0007_0.0000_0.0001/y2.txt};
	\addplot[red,semithick,densely dashed] file {wykresy/dmc_200_200_200/1.0000_1.0000_1.0000_1.0000_1.0000_1.0000_1.0000/yzad2.txt};
	\legend{$Nasze$,$Obliczone$,$y2_{zad}$}
	\nextgroupplot
	[
	xmin=0,xmax=910,ymin=-5,ymax=5,
	xlabel={$k$},
	ylabel={$y3$},
	xtick={0,100,200,300,400,100,600,700,800,900},
	ytick={-5,-2.5,0,2.5,5},
	y tick label style={/pgf/number format/1000 sep=},
	legend pos=south east,
	] 
	\addplot[blue,semithick] file {wykresy/dmc_200_200_200/2.7000_40.0000_7.0000_20.0000_1.0000_0.0000_0.0000/y3.txt};
	\addplot[green,semithick] file {wykresy/dmc_200_200_200/6.7137_40.1488_7.3852_20.0812_0.0007_0.0000_0.0001/y3.txt};
	\addplot[red,semithick,densely dashed] file {wykresy/dmc_200_200_200/1.0000_1.0000_1.0000_1.0000_1.0000_1.0000_1.0000/yzad3.txt};
	\legend{$Nasze$,$Obliczone$,$y3_{zad}$}
	\end{groupplot}
	\end{tikzpicture}
	\caption{Por�wnanie wyliczonych i optymalnych nastaw dla DMC - wyj�cia}
	\label{DMCw}
\end{figure}
\FloatBarrier
\section{Wnioski}
Optymalizacja wska�nika jako�ci poprawi�a przebiegi wszystkich regulator�w. Regulacja uzyskana z u�yciem regulatora DMC uzyska�a definitywnie lepsze wyniki ni� kt�rykolwiek PID, je�li chodzi o b��d regulacji i wygl�d przebieg�w wyj��. Mimo to tak gwa�towny regulator m�g�by by� niemo�liwy do wykorzystania w realnym �yciu ze wzgl�du na du�e wachania sterowa�.
%\chapter{Zadanie 6: Nieliniowy regulator DMC}

\section{Wyb�r parametru D funkcji przynale�no�ci}
B��d - 3,8294e+03
\begin{figure}[tb]
	\centering
	\begin{tikzpicture}
	\begin{groupplot}[group style={group size=1 by 2,vertical sep={2 cm}},
	width=0.9\textwidth,height=0.4\textwidth]
	\nextgroupplot
	[
	xmin=0,xmax=1500,ymin=-1,ymax=1,
	xlabel={$k$},
	ylabel={$U$},
	xtick={0,250,500,750,1000,1250,1500},
	ytick={-1,0,1},
	y tick label style={/pgf/number format/1000 sep=},
	]
	\addplot[blue,semithick] file {wykresy/zad6/p6dmcUd10.txt};
	\nextgroupplot
	[
	xmin=0,xmax=1500,ymin=-1,ymax=6,
	xlabel={$k$},
	ylabel={$Y$},
	xtick={0,250,500,750,1000,1250,1500},
	ytick={-1,0,1,2,3,4,5,6},
	y tick label style={/pgf/number format/1000 sep=},
	legend pos=south east,
	]
	\addplot[blue,semithick] file {wykresy/zad6/p6dmcYd10.txt};
	\addplot[orange,semithick,densely dashed] file {wykresy/zad6/lab6Yzad.txt};
	\end{groupplot}
	\end{tikzpicture}
	\caption{d = 10}
	\label{D1}
\end{figure}
\FloatBarrier

B��d - 553,4929
\begin{figure}[tb]
	\centering
	\begin{tikzpicture}
	\begin{groupplot}[group style={group size=1 by 2,vertical sep={2 cm}},
	width=0.9\textwidth,height=0.4\textwidth]
	\nextgroupplot
	[
	xmin=0,xmax=1500,ymin=-1,ymax=1,
	xlabel={$k$},
	ylabel={$U$},
	xtick={0,250,500,750,1000,1250,1500},
	ytick={-1,0,1},
	y tick label style={/pgf/number format/1000 sep=},
	]
	\addplot[blue,semithick] file {wykresy/zad6/p6dmcUd04.txt};
	\nextgroupplot
	[
	xmin=0,xmax=1500,ymin=-1,ymax=6,
	xlabel={$k$},
	ylabel={$Y$},
	xtick={0,250,500,750,1000,1250,1500},
	ytick={-1,0,1,2,3,4,5,6},
	y tick label style={/pgf/number format/1000 sep=},
	legend pos=south east,
	]
	\addplot[blue,semithick] file {wykresy/zad6/p6dmcYd04.txt};
	\addplot[orange,semithick,densely dashed] file {wykresy/zad6/lab6Yzad.txt};
	\end{groupplot}
	\end{tikzpicture}
	\caption{d = 0,4}
	\label{D2}
\end{figure}
\FloatBarrier
\section{Dwa regulatory lokalne}
B��d - 568,1155
\begin{figure}[tb]
	\centering
	\begin{tikzpicture}
	\begin{groupplot}[group style={group size=1 by 2,vertical sep={2 cm}},
	width=0.9\textwidth,height=0.4\textwidth]
	\nextgroupplot
	[
	xmin=0,xmax=1500,ymin=-1,ymax=1,
	xlabel={$k$},
	ylabel={$U$},
	xtick={0,250,500,750,1000,1250,1500},
	ytick={-1,0,1},
	y tick label style={/pgf/number format/1000 sep=},
	]
	\addplot[blue,semithick] file {wykresy/zad6/p6dmcUdreg2.txt};
	\nextgroupplot
	[
	xmin=0,xmax=1500,ymin=-1,ymax=6,
	xlabel={$k$},
	ylabel={$Y$},
	xtick={0,250,500,750,1000,1250,1500},
	ytick={-1,0,1,2,3,4,5,6},
	y tick label style={/pgf/number format/1000 sep=},
	legend pos=south east,
	]
	\addplot[blue,semithick] file {wykresy/zad6/p6dmcYdreg2.txt};
	\addplot[orange,semithick,densely dashed] file {wykresy/zad6/lab6Yzad.txt};
	\end{groupplot}
	\end{tikzpicture}
	\caption{Dwa regulatory lokalne}
	\label{reg2dmc}
\end{figure}
\FloatBarrier
\section{Trzy regulatory lokalne}
B��d - 565,2050
\begin{figure}[tb]
	\centering
	\begin{tikzpicture}
	\begin{groupplot}[group style={group size=1 by 2,vertical sep={2 cm}},
	width=0.9\textwidth,height=0.4\textwidth]
	\nextgroupplot
	[
	xmin=0,xmax=1500,ymin=-1,ymax=1,
	xlabel={$k$},
	ylabel={$U$},
	xtick={0,250,500,750,1000,1250,1500},
	ytick={-1,0,1},
	y tick label style={/pgf/number format/1000 sep=},
	]
	\addplot[blue,semithick] file {wykresy/zad6/p6dmcUdreg3.txt};
	\nextgroupplot
	[
	xmin=0,xmax=1500,ymin=-1,ymax=6,
	xlabel={$k$},
	ylabel={$Y$},
	xtick={0,250,500,750,1000,1250,1500},
	ytick={-1,0,1,2,3,4,5,6},
	y tick label style={/pgf/number format/1000 sep=},
	legend pos=south east,
	]
	\addplot[blue,semithick] file {wykresy/zad6/p6dmcYdreg3.txt};
	\addplot[orange,semithick,densely dashed] file {wykresy/zad6/lab6Yzad.txt};
	\end{groupplot}
	\end{tikzpicture}
	\caption{Trzy regulatory lokalne}
	\label{reg3dmc}
\end{figure}
\FloatBarrier
\section{Cztery regulatory lokalne}

B��d - 553,4929
\begin{figure}[tb]
	\centering
	\begin{tikzpicture}
	\begin{groupplot}[group style={group size=1 by 2,vertical sep={2 cm}},
	width=0.9\textwidth,height=0.4\textwidth]
	\nextgroupplot
	[
	xmin=0,xmax=1500,ymin=-1,ymax=1,
	xlabel={$k$},
	ylabel={$U$},
	xtick={0,250,500,750,1000,1250,1500},
	ytick={-1,0,1},
	y tick label style={/pgf/number format/1000 sep=},
	]
	\addplot[blue,semithick] file {wykresy/zad6/p6dmcUd04.txt};
	\nextgroupplot
	[
	xmin=0,xmax=1500,ymin=-1,ymax=6,
	xlabel={$k$},
	ylabel={$Y$},
	xtick={0,250,500,750,1000,1250,1500},
	ytick={-1,0,1,2,3,4,5,6},
	y tick label style={/pgf/number format/1000 sep=},
	legend pos=south east,
	]
	\addplot[blue,semithick] file {wykresy/zad6/p6dmcYd04.txt};
	\addplot[orange,semithick,densely dashed] file {wykresy/zad6/lab6Yzad.txt};
	\end{groupplot}
	\end{tikzpicture}
	\caption{Cztery regulatory lokalne}
	\label{reg4dmc}
\end{figure}
\FloatBarrier

\section{Pi�� regulator�w lokalnych}

B��d - 551,9820
\begin{figure}[tb]
	\centering
	\begin{tikzpicture}
	\begin{groupplot}[group style={group size=1 by 2,vertical sep={2 cm}},
	width=0.9\textwidth,height=0.4\textwidth]
	\nextgroupplot
	[
	xmin=0,xmax=1500,ymin=-1,ymax=1,
	xlabel={$k$},
	ylabel={$U$},
	xtick={0,250,500,750,1000,1250,1500},
	ytick={-1,0,1},
	y tick label style={/pgf/number format/1000 sep=},
	]
	\addplot[blue,semithick] file {wykresy/zad6/p6dmcUdreg5.txt};
	\nextgroupplot
	[
	xmin=0,xmax=1500,ymin=-1,ymax=6,
	xlabel={$k$},
	ylabel={$Y$},
	xtick={0,250,500,750,1000,1250,1500},
	ytick={-1,0,1,2,3,4,5,6},
	y tick label style={/pgf/number format/1000 sep=},
	legend pos=south east,
	]
	\addplot[blue,semithick] file {wykresy/zad6/p6dmcYdreg5.txt};
	\addplot[orange,semithick,densely dashed] file {wykresy/zad6/lab6Yzad.txt};
	\end{groupplot}
	\end{tikzpicture}
	\caption{Pi�� regulator�w lokalnych}
	\label{reg5dmc}
\end{figure}
\FloatBarrier
\chapter{Laboratorium: Zadanie 7: Rozmyty regulator DMC}
\section{Teoria}

Spos�b wyznaczania sterowania nie r�ni si� od tego opisanego w poprzednim podpunkcie; wci�� w ka�dej iteracji rozmytego regulatora obliczane jest sterowanie dla ka�dego regulatora z przyj�tych przedzia��w i obliczane jest ko�cowe sterowanie bior�c pod uwag� stopie� przynale�no�ci aktualnego Y(k) do funkcji opisuj�cych ka�dy z tych przedzia��w.

Wci�� jest to ten sam obiekt, kt�rego charakterystyka sk�ada si� z dw�ch liniowych regulator�w, nie ma wi�c potrzeby implementowania wi�kszej ilo�ci regulator�w lokalnych ani zmiany postaci funkcji przynale�no�ci.

\subsection{Odpowiedzi skokowe}
Obydwa regulatory lokalne powinny korzysta� z odpowiadaj�cych im odpowiedzi skokowych. Aby odpowiedzi te popranie reprezentowa�y charakter przedzia�u, w kt�rym pracuje dany regulator lokalny, ich punkty pocz�tkowe i ko�cowe musz� si� zawiera� w tych przedzia�ach. 
Do algorytmu regulatora dolnego zosta�a wykorzystana znormalizowana odpowied� skokowa z $U = 36$ o $dU = 5$, natomiast dla g�rnego przedzia�u znormalizowana odpowied� dla takiego samego skoku ale z $U = 55$. 

\section{Dob�r regulator�w lokalnych}
\subsection{Pocz�tkowe nastawy}


\begin{figure}[tb]
	\centering
	\begin{tikzpicture}
	\begin{groupplot}[group style={group size=1 by 2,vertical sep={2 cm}},
	width=0.9\textwidth,height=0.5\textwidth]
	\nextgroupplot
	[
	xmin=0,xmax=1210,ymin=-5,ymax=105,
	xlabel={$k$},
	ylabel={$U(k)$},
	xtick={0,100,200,300,400,500,600,700,800,900,1000,1100,1200},
	ytick={0,10,20,30,40,50,60,70,80,90,100},
	y tick label style={/pgf/number format/1000 sep=},
	legend pos=south east,
	]
	\addplot[blue,semithick] file {wykresy/lab6dmcU1_fail.txt};
	\nextgroupplot
	[
	xmin=0,xmax=1210,ymin=36,ymax=47,
	xlabel={$k$},
	ylabel={$Y(k)$},
	xtick={0,100,200,300,400,500,600,700,800,900,1000,1100,1200},
	ytick={36,37,38,39,40,41,42,43,44,45,46},
	y tick label style={/pgf/number format/1000 sep=},
	legend pos=north east,
	]
	\addplot[red,semithick] file{wykresy/lab6dmcY1_fail.txt}; 
	\addplot[orange,semithick,densely dashed] file{wykresy/lab6Yzad.txt}; 
	\legend{$Y$,$Y^{zad}$,$Y_2$,}
	\end{groupplot}
	\end{tikzpicture}
	\caption{Dzia�anie rozmytego regulatora z dwoma lokalnymi regulatorami PID o  nastawach  $K_p^1 = 5, T_i^1 = 75, T_d^1 = 1.25$,  $K_p^2 = 7, T_i^2 = 45, T_d^2 = 1$} 
	\label{rozmytydmc1_fail}
\end{figure}
\FloatBarrier	
	
	\begin{figure}[tb]
		\centering
		\begin{tikzpicture}
		\begin{groupplot}[group style={group size=1 by 2,vertical sep={2 cm}},
		width=0.9\textwidth,height=0.5\textwidth]
		\nextgroupplot
		[
		xmin=0,xmax=1210,ymin=-5,ymax=105,
		xlabel={$k$},
		ylabel={$U(k)$},
		xtick={0,100,200,300,400,500,600,700,800,900,1000,1100,1200},
		ytick={0,10,20,30,40,50,60,70,80,90,100},
		y tick label style={/pgf/number format/1000 sep=},
		legend pos=south east,
		]
		\addplot[blue,semithick] file {wykresy/lab6dmcU1_fail2.txt};
		\nextgroupplot
		[
		xmin=0,xmax=1210,ymin=36,ymax=47,
		xlabel={$k$},
		ylabel={$Y(k)$},
		xtick={0,100,200,300,400,500,600,700,800,900,1000,1100,1200},
		ytick={36,37,38,39,40,41,42,43,44,45,46},
		y tick label style={/pgf/number format/1000 sep=},
		legend pos=north east,
		]
		\addplot[red,semithick] file{wykresy/lab6dmcY1_fail2.txt}; 
		\addplot[orange,semithick,densely dashed] file{wykresy/lab6Yzad.txt}; 
		\legend{$Y$,$Y^{zad}$,$Y_2$,}
		\end{groupplot}
		\end{tikzpicture}
		\caption{Dzia�anie rozmytego regulatora z dwoma lokalnymi regulatorami PID o  nastawach  $K_p^1 = 5, T_i^1 = 75, T_d^1 = 1.25$,  $K_p^2 = 7, T_i^2 = 45, T_d^2 = 1$} 
		\label{rozmytydmc1_fail2}
	
\end{figure}
\FloatBarrier
	\begin{figure}[tb]
	\centering
	\begin{tikzpicture}
	\begin{groupplot}[group style={group size=1 by 2,vertical sep={2 cm}},
	width=0.9\textwidth,height=0.5\textwidth]
	\nextgroupplot
	[
	xmin=0,xmax=1210,ymin=-5,ymax=105,
	xlabel={$k$},
	ylabel={$U(k)$},
	xtick={0,100,200,300,400,500,600,700,800,900,1000,1100,1200},
	ytick={0,10,20,30,40,50,60,70,80,90,100},
	y tick label style={/pgf/number format/1000 sep=},
	legend pos=south east,
	]
	\addplot[blue,semithick] file {wykresy/lab6dmcU2.txt};
	\nextgroupplot
	[
	xmin=0,xmax=1210,ymin=36,ymax=47,
	xlabel={$k$},
	ylabel={$Y(k)$},
	xtick={0,100,200,300,400,500,600,700,800,900,1000,1100,1200},
	ytick={36,37,38,39,40,41,42,43,44,45,46},
	y tick label style={/pgf/number format/1000 sep=},
	legend pos=north east,
	]
	\addplot[red,semithick] file{wykresy/lab6dmcY2.txt}; 
	\addplot[orange,semithick,densely dashed] file{wykresy/lab6Yzad.txt}; 
	\legend{$Y$,$Y^{zad}$,$Y_2$,}
	\end{groupplot}
	\end{tikzpicture}
	\caption{Dzia�anie rozmytego regulatora z dwoma lokalnymi regulatorami PID o  nastawach  $K_p^1 = 5, T_i^1 = 75, T_d^1 = 1.25$,  $K_p^2 = 7, T_i^2 = 45, T_d^2 = 1$} 
	\label{rozmytydmc2}
\end{figure}
\FloatBarrier

\begin{figure}[tb]
\centering
\begin{tikzpicture}
\begin{groupplot}[group style={group size=1 by 2,vertical sep={2 cm}},
width=0.9\textwidth,height=0.5\textwidth]
\nextgroupplot
[
xmin=0,xmax=1210,ymin=-5,ymax=105,
xlabel={$k$},
ylabel={$U(k)$},
xtick={0,100,200,300,400,500,600,700,800,900,1000,1100,1200},
ytick={0,10,20,30,40,50,60,70,80,90,100},
y tick label style={/pgf/number format/1000 sep=},
legend pos=south east,
]
\addplot[blue,semithick] file {wykresy/lab6dmcU3.txt};
\nextgroupplot
[
xmin=0,xmax=1210,ymin=36,ymax=47,
xlabel={$k$},
ylabel={$Y(k)$},
xtick={0,100,200,300,400,500,600,700,800,900,1000,1100,1200},
ytick={36,37,38,39,40,41,42,43,44,45,46},
y tick label style={/pgf/number format/1000 sep=},
legend pos=north east,
]
\addplot[red,semithick] file{wykresy/lab6dmcY3.txt}; 
\addplot[orange,semithick,densely dashed] file{wykresy/lab6Yzad.txt}; 
\legend{$Y$,$Y^{zad}$,$Y_2$,}
\end{groupplot}
\end{tikzpicture}
\caption{Dzia�anie rozmytego regulatora z dwoma lokalnymi regulatorami PID o  nastawach  $K_p^1 = 5, T_i^1 = 75, T_d^1 = 1.25$,  $K_p^2 = 7, T_i^2 = 45, T_d^2 = 1$} 
\label{rozmytydmc3}
\end{figure}
\FloatBarrier
%\input{lab3_opisobiektu}
%\input{lab3_zad1}
%\input{lab3_zad2}
%\input{lab3_zad3}
%\input{lab3_zad4}
%\input{lab3_zad5}
%\input{lab3_zad5DMC}
\end{document}


