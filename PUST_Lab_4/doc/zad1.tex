\chapter{Zadanie 1: Punkt pracy}
Pierwszym poleceniem by�o okre�lenie warto�ci wyj�cia obiektu $Y_{pp}$ (pomiaru $T1$) w punkcie pracy $U_{pp} = 36$. Osi�gn�li�my j� ustawiaj�c warto�� sterowania (moc grzania grza�ki $G1$) na $U_{pp}$ i odczekuj�c znaczn� ilo�� czasu (powy�ej 5 minut). Jak wida� na wykresie \ref{pp}, wyj�cie ustabilizowa�o si� w okolicy 35 stopni Celcjusza. Jendak�e, z powodu narastaj�cej temperatury w ci�gu zaj��, punkt pracy zmienia� si�, nawet o ponad stopie� w g�r�, co b�dzie widoczne w nast�pnych zadaniach.


\begin{figure}[tb]
	\centering
	\begin{tikzpicture}
	\begin{axis}[
	legend pos=south east,
	width=0.9\textwidth,
	xmin=0,xmax=300,ymin=25,ymax=37,
	xlabel={$k$},
	ylabel={$S(k)$},
	xtick={0,50,100,150,200,250,300},
	ytick={25,26,27,28,29,30,31,32,33,34,35,36,37},
	y tick label style={/pgf/number format/1000 sep=},
	]
	\addplot[blue,semithick] file {wykresy/lab1U.txt};
	\addplot[red,semithick] file {wykresy/lab1Y.txt};
	
	\legend{$U$,$Y$}
	\end{axis}
	\end{tikzpicture}
	\caption{Zachowanie obiektu w punkcie pracy, $U_{pp}=\num{36}$}
	\label{pp}
\end{figure}
\FloatBarrier