\chapter{Zadanie 2: Odpowiedzi skokowe}
\section{Odpowiedzi skokowe}

W tej cz�ci projektu nale�a�o wyznaczy� symulacyjnie odpowiedzi skokowe (rys. \ref{odpskok}). Eksperyment zak�ada�, i� obiekt b�dzie na pocz�tku w punkcie pracy, a nast�pnie w chwili $k=15$ zostanie wykonany skok jednostkowy. 
\begin{figure}[tb]
	\centering
	\begin{tikzpicture}
	\begin{axis}[
	width=0.9\textwidth,
	height=0.3\textheight,
	xmin=0,xmax=150,ymin=-1,ymax=1,
	xlabel={$k$},
	ylabel={$U$},
	xtick={0,50,100,150},
	ytick={-1,-0.5,0,0.5,1},
	]
	\addplot[blue,semithick] file {wykresy/zad2/z2_u1.txt};
	\addplot[red,semithick] file {wykresy/zad2/z2_u4.txt};
	\addplot[green,semithick] file {wykresy/zad2/z2_u7.txt};
	\addplot[yellow,semithick] file {wykresy/zad2/z2_u10.txt};
	\addplot[brown,semithick] file {wykresy/zad2/z2_u13.txt};
	\addplot[orange,semithick] file {wykresy/zad2/z2_u16.txt};
	\addplot[magenta,semithick] file {wykresy/zad2/z2_u19.txt};
	\end{axis}
	\end{tikzpicture}
	\caption{Sterowanie}
	\label{uskok}
\end{figure}

\begin{figure}[tb]
	\centering
	\begin{tikzpicture}
	\begin{axis}[
	width=0.9\textwidth,
	height=0.3\textheight,
	xmin=0,xmax=200,ymin=-0.5,ymax=6,
	xlabel={$k$},
	ylabel={$Y$},
	xtick={0,50,100,150,200},
	ytick={-0.5,0,0.5,1,1.5,2,2.5,3,3.5,4,4.5,5,5.5,6},
	]
	\addplot[blue,semithick] file {wykresy/zad2/z2_y1.txt};
	\addplot[red,semithick] file {wykresy/zad2/z2_y4.txt};
	\addplot[green,semithick] file {wykresy/zad2/z2_y7.txt};
	\addplot[yellow,semithick] file {wykresy/zad2/z2_y10.txt};
	\addplot[brown,semithick] file {wykresy/zad2/z2_y13.txt};
	\addplot[orange,semithick] file {wykresy/zad2/z2_y16.txt};
	\addplot[magenta,semithick] file {wykresy/zad2/z2_y19.txt};
	\end{axis}
	\end{tikzpicture}
	\caption{Wyj�cie}
	\label{yskok}
\end{figure}


\section{Charakterystyka statyczna}
Poni�ej zosta�a zaprezentowana charakterystyka statyczna procesu $y(u)$ (rys. \ref{stat}).
Na podstawie zawartego wykresu mo�na wywnioskowa�, �e w�a�ciwo�ci statyczne procesu s� nieliniowe. 

\begin{figure}[tb]
	\centering
	\begin{tikzpicture}
	\begin{axis}[
	width=0.9\textwidth,
	height=0.5\textheight,
	xmin=-1,xmax=1,ymin=-0.5,ymax=6,
	xlabel={$U$},
	ylabel={$Y$},
	xtick={-1,-0.8,-0.6,-0.4,-0.2,0,0.2,0.4,0.6,0.8,1},
	ytick={-0.5,0,0.5,1,1.5,2,2.5,3,3.5,4,4.5,5,5.5,6},
	]
	\addplot[blue,semithick] file {wykresy/zad2/z2_stat.txt};
	\end{axis}
	\end{tikzpicture}
	\caption{Charakterystyka statyczna}
	\label{stat}
\end{figure}

\section{Charakterystyka dynamiczna}
Charakterystyka dynamiczna zosta�a wyznaczona zale�nie od wielko�ci skoku sterowania. Zmierzone zosta�o, po ilu krokach od momentu skoku r�nica warto�ci wyj�� obiektu i punktu pracy $Y_{pp}$ wynosi�a powy�ej $90\%$ ca�kowitego skoku warto�ci  wyj�� obiektu $Y(k)$. Z otrzymanych danych wynika, �e charakterystyka dynamiczna jest nieliniowa (rys. \ref{dynamiczna}).
\begin{figure}[tb]
	\centering
	\begin{tikzpicture}
	\begin{axis}[
	width=0.9\textwidth,
	height=0.5\textheight,
	xmin=-1,xmax=1,ymin=0,ymax=18,
	xlabel={$U$},
	ylabel={$Y$},
	xtick={-1,-0.8,-0.6,-0.4,-0.2,0,0.2,0.4,0.6,0.8,1},
	ytick={0,2,4,6,8,10,12,14,16,18},
	]
	\addplot[blue,semithick] file {wykresy/zad2/z2_dyn.txt};
	\end{axis}
	\end{tikzpicture}
	\caption{Charakterystyka dynamiczna}
	\label{dynamiczna}
\end{figure}