\chapter{Zadanie 2: Odpowied� skokowa}


\begin{figure}[tb]
\centering
\begin{tikzpicture}
\begin{axis}[
width=0.9\textwidth,
xmin=0,xmax=350,ymin=36,ymax=44,
xlabel={$k$},
ylabel={$S(k)$},
xtick={0,50,100,150,200,250,300,350},
ytick={36,36.5,38,38.5,40,40.5,42,42.5,44},
y tick label style={/pgf/number format/1000 sep=},
]
\addplot[blue,semithick] file {wykresy/skokY_5.txt};
\addplot[green,semithick] file {wykresy/skokY_10.txt};
\addplot[magenta,semithick] file {wykresy/skokY_15.txt};
\addplot[orange,semithick] file {wykresy/skokY_0.txt};

\legend{$dU=\num{5}$,$dU=\num{10}$,$dU=\num{15}$,$Y_{pp}=\num{36.5}$}
\end{axis}
\end{tikzpicture}
\caption{Wykres S(k) dla r�nych skok�w sterowania z Upp=\num{36} o dU}
\label{skok}
\end{figure}


