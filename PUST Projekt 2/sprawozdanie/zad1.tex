\chapter{Zadanie 1: Punkt pracy}
Pierwszym poleceniem by�o zweryfikowanie poprawno�ci punktu pracy obiektu. Uda�o si� to osi�gn�� za pomoc� prostego sprawdzenia, przy jakiej warto�ci wyj�cia stabilizuje si� obiekt przy sta�ym sterowaniu oraz sta�ym zak��ceniu, r�wnym ich warto�ciom w punkcie pracy ($U_{pp}=0$, $Z_{pp}=0$). Eksperyment potwierdzi� wcze�niej podan� warto�� wyj�cia ($Y_{pp}=0$), a jego przebieg
obrazuje wykres \ref{fig:punktpracy}.

\begin{figure}[tb]
	\centering
	\begin{tikzpicture}
	\begin{groupplot}[group style={group size=1 by 3,vertical sep={2 cm}},
	width=0.9\textwidth,height=0.3\textwidth]
	\nextgroupplot
	[
	xmin=0,xmax=200,ymin=-1,ymax=1,
	xlabel={$k$},
	ylabel={$U$},
	xtick={0,50,100,150,200},
	ytick={-1,-0.5,0,0.5,1},
	y tick label style={/pgf/number format/1000 sep=},
	]
	\addplot[blue,semithick] file {wykresy/z1u.txt};
	\nextgroupplot
	[
	xmin=0,xmax=200,ymin=-1,ymax=1,
	xlabel={$k$},
	ylabel={$Z$},
	xtick={0,50,100,150,200},
	ytick={-1,-0.5,0,0.5,1},
	y tick label style={/pgf/number format/1000 sep=},
	legend pos=south east,
	]
	\addplot[green,semithick] file {wykresy/z1z.txt};
	\nextgroupplot
	[
	xmin=0,xmax=200,ymin=-1,ymax=1,
	xlabel={$k$},
	ylabel={$Y$},
	xtick={0,50,100,150,200},
	ytick={-1,-0.5,0,0.5,1},
	y tick label style={/pgf/number format/1000 sep=},
	legend pos=south east,
	]
	\addplot[red,semithick] file {wykresy/z1y.txt};
	\end{groupplot}
	\end{tikzpicture}
	\caption{Zachowanie obiektu w punkcie pracy}
	\label{fig:punktpracy}
\end{figure}