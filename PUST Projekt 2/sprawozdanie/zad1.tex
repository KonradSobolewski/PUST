\chapter{Zadanie 1: Punkt pracy}
Pierwszym poleceniem by�o zweryfikowanie poprawno�ci punktu pracy obiektu. Uda�o si� to osi�gn�� za pomoc� prostego sprawdzenia dla jakiej
warto�ci wyj�cia stabilizuje si� obiekt przy sta�ym sterowaniu r�wnym sterowaniu w punkcie pracy.Eksperyment potwierdzi� wcze�niej opisane warto�ci, a jego przebieg
obrazuje wykres \ref{fig:z1}.

\begin{figure}[tb]
\centering
\begin{tikzpicture}
\begin{axis}[
width=0.9\textwidth,
xmin=0,xmax=200,ymin=0,ymax=2.5,
xlabel={$k$},
ylabel={$Y(k)$},
xtick={0,50,100,150,200},
ytick={0,0.5,1,1.5,2.0,2.5},
y tick label style={/pgf/number format/1000 sep=},
]
\addplot[blue,semithick] file {wykresy/z1.txt};
\legend{$Y(k)$}
\end{axis}
\end{tikzpicture}
\caption{Wykres Y(k) d���cy do Ypp=2.5 dla sta�ego sterowanie r�wnego Upp=\num{1.0}}
\label{fig:z1}
\end{figure}