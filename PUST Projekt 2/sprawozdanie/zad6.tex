\chapter{Zadanie 6: Algorytm przy zak��ceniu zmiennym sinusoidalne}

W tym zadaniu zak��cenia maj� charakter sygna�u sinusoidalnie zmiennego.
Poni�szy wykres przedstawia przebiegi wej�cia i wyj�cia obiektu sterowanego przy pomocy regulatgora DMC w wersji z kolejno nieuwzgl�dnieniem ($U(k), Y(k)$) i uwzgl�dnieniem ($Uz(k), Yz(k)$) mierzonych zak��ce�.

\begin{figure}[tb]
	\centering
	\begin{tikzpicture}
	\begin{groupplot}[group style={group size=1 by 3,vertical sep={2 cm}},
	width=0.9\textwidth,height=0.3\textwidth]
	\nextgroupplot
	[
	xmin=0,xmax=150,ymin=-0.5,ymax=2,
	xlabel={$k$},
	ylabel={$U$},
	xtick={0,25,50,75,100,125,150},
	ytick={-0.5,0,0.5,1,1.5,2},
	y tick label style={/pgf/number format/1000 sep=},
	]
	\addplot[green,semithick] file {wykresy/6dmcu_8_1_0.3000_40_0.txt};
	\addplot[blue,semithick] file {wykresy/6dmcu_8_1_0.3000_40_1.txt};
		\legend{$U(k)$,$Uz(k)$,}
	\nextgroupplot
	[
	xmin=0,xmax=150,ymin=0,ymax=2,
	xlabel={$k$},
	ylabel={$Y$},
	xtick={0,25,50,75,100,125,150},
	ytick={0,0.5,1,1.5,2},
	y tick label style={/pgf/number format/1000 sep=},
	legend pos=south east,
	]
	\addplot[green,semithick] file {wykresy/6dmcy_8_1_0.3000_40_0.txt};
	\addplot[blue,semithick] file {wykresy/6dmcy_8_1_0.3000_40_1.txt};
	\addplot[orange,semithick] file {wykresy/dmcyzad_1.0000.txt};
	\legend{$Y(k)$,$Yz(k)$,$Y^{zad}(k)$}
	\nextgroupplot
	[
	xmin=0,xmax=150,ymin=-0.6,ymax=0.6,
	xlabel={$k$},
	ylabel={$Z$},
	xtick={0,50,100,150},
	ytick={-0.5,0,0.5},
	y tick label style={/pgf/number format/1000 sep=},
	legend pos=south east,
	]
	\addplot[red,semithick] file {wykresy/6z.txt};
	\end{groupplot}
	\end{tikzpicture}
	\caption{Por�wnanie regulacji z i bez uwzgl�dniania zak��ce� w algorytmie sterowania}
	\label{porbezZiZ}
\end{figure}
\FloatBarrier

Wska�nik regulacji, gdy zak��cenia nie s� brane pod uwag�: $E = 18.3737$
\

Wska�nik regulacji, gdy zak��cenia s� brane pod uwag�: $E = 8.4010$
\\

Jak wida�, mimo �e sinusoidalne zak��cenia w obu przypadkach uniemo�liwiaj� akceptowaln� jako�� regulacji, to uwzgl�dnienie tych zak��ce� pozwala na znacz�ce zmieniejszenie b��du.


