\chapter{Zadanie 3: Znormalizowane odpowiedzi skokowe}
\label{sec:zad3}
Normalizacja odpowiedzi skokowej polega na przesuni�ciu ka�dej warto�ci wyj�cia obiektu o warto�� w punkcie pracy, a nast�pnie podzielenie jej przez d�ugo�� skoku sterowania (dla toru sterowania)/zak��cenia (dla toru zak��cenia).
\begin{equation}
S=\frac{Y-Ypp}{dU}
\end{equation}
\begin{equation}
S_z=\frac{Y-Ypp}{dZ}
\end{equation}
W ten spos�b otrzymujemy warto�ci odpowiedzi skokowych w formie w jakiej otrzymaliby�my je robi�c skok jednostkowy na sterowaniu/zak��ceniu. Takie odpowiedzi skokowe s� gotowe do u�ytku w regulatorze $DMC$. Obie odpowiedzi zosta�y przedstawione na poni�szych wykresach \ref{fig:z3s} oraz \ref{fig:z3sz}.
\begin{figure}[tb]
	\centering
	\begin{tikzpicture}
	\begin{axis}[
	width=0.9\textwidth,
	xmin=0,xmax=185,ymin=0,ymax=2.6,
	xlabel={$k$},
	ylabel={$s$},
	xtick={0,50,100,150},
	ytick={0,0.5,1,1.5,2,2,5},
	y tick label style={/pgf/number format/1000 sep=},
	]
	\addplot[blue,semithick] file {wykresy/z3s.txt};
	\end{axis}
	\end{tikzpicture}
	\caption{Wykres znormalizowanej odpowiedzi skokowej toru U}
	\label{fig:z3s}
\end{figure}

\begin{figure}[tb]
	\centering
	\begin{tikzpicture}
	\begin{axis}[
	width=0.9\textwidth,
	xmin=0,xmax=185,ymin=0,ymax=2,
	xlabel={$k$},
	ylabel={$s_z$},
	xtick={0,50,100,150},
	ytick={0,0.5,1,1.5,2},
	y tick label style={/pgf/number format/1000 sep=},
	]
	\addplot[blue,semithick] file {wykresy/z3sz.txt};
	\end{axis}
	\end{tikzpicture}
	\caption{Wykres znormalizowanej odpowiedzi skokowej toru Z}
	\label{fig:z3sz}
\end{figure}